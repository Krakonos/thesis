%%% The main file. It contains definitions of basic parameters and includes all other parts.

%% Settings for single-side (simplex) printing
% Margins: left 40mm, right 25mm, top and bottom 25mm
% (but beware, LaTeX adds 1in implicitly)
\documentclass[12pt,a4paper,final]{report}
\setlength\textwidth{145mm}
\setlength\textheight{247mm}
\setlength\oddsidemargin{15mm}
\setlength\evensidemargin{15mm}
\setlength\topmargin{0mm}
\setlength\headsep{0mm}
\setlength\headheight{0mm}
% \openright makes the following text appear on a right-hand page
\let\openright=\clearpage

%% Settings for two-sided (duplex) printing
% \documentclass[12pt,a4paper,twoside,openright]{report}
% \setlength\textwidth{145mm}
% \setlength\textheight{247mm}
% \setlength\oddsidemargin{14.2mm}
% \setlength\evensidemargin{0mm}
% \setlength\topmargin{0mm}
% \setlength\headsep{0mm}
% \setlength\headheight{0mm}
% \let\openright=\cleardoublepage

%% Character encoding: usually latin2, cp1250 or utf8:
\usepackage[utf8]{inputenc}

%% Further useful packages (included in most LaTeX distributions)
\usepackage{amsmath}        % extensions for typesetting of math
\usepackage{amsfonts}       % math fonts
\usepackage{amsthm}         % theorems, definitions, etc.
\usepackage{bbding}         % various symbols (squares, asterisks, scissors, ...)
\usepackage{bm}             % boldface symbols (\bm)
\usepackage{graphicx}       % embedding of pictures
\usepackage{fancyvrb}       % improved verbatim environment
%\usepackage{natbib}         % citation style AUTHOR (YEAR), or AUTHOR [NUMBER]
\usepackage[nottoc]{tocbibind} % makes sure that bibliography and the lists
			    % of figures/tables are included in the table
			    % of contents
\usepackage{dcolumn}        % improved alignment of table columns
\usepackage{booktabs}       % improved horizontal lines in tables
\usepackage{paralist}       % improved enumerate and itemize
\usepackage[usenames]{xcolor}  % typesetting in color
%\usepackage[czech]{babel}
%\usepackage[toc,page]{appendix}

\usepackage{tcolorbox}
\usepackage{listings}
\usepackage{wrapfig}
\usepackage{caption}
\usepackage{subcaption}

%\usepackage{underscore}

\usepackage[backend=bibtex,style=alphabetic,firstinits=true,block=space]{biblatex}
\bibliography{bibliography.bib}
\nocite{*}
\setcounter{biburllcpenalty}{0}
\setcounter{biburlucpenalty}{0}


\usepackage{tikz} % To generate the plot from csv
\usepackage{pgfplots}
\usepgfplotslibrary{dateplot}

%%% Basic information on the thesis

% Thesis title in English (exactly as in the formal assignment)
\def\ThesisTitle{Scalable link-time optimization}
\def\CzThesisTitle{Škálovatelná optimalizace celých programů}

% Author of the thesis
\def\ThesisAuthor{Ladislav Láska}

% Year when the thesis is submitted
\def\YearSubmitted{2017}

% Name of the department or institute, where the work was officially assigned
% (according to the Organizational Structure of MFF UK in English,
% or a full name of a department outside MFF)
\def\Department{ Computer Science Institute of Charles University }
\def\CzDepartment{ Informatický Ústav Univerzity Karlovy }


% Is it a department (katedra), or an institute (ústav)?
\def\DeptType{Institute}
\def\CzDeptType{Ústav}

% Thesis supervisor: name, surname and titles
\def\Supervisor{Mgr. Jan Hubička Ph.D.}

% Supervisor's department (again according to Organizational structure of MFF)
\def\SupervisorsDepartment{Computer Science Institute of Charles University}
\def\CzSupervisorsDepartment{Informatický Ústav Univerzity Karlovy}

% Study programme and specialization
\def\StudyProgramme{Informatics}
\def\StudyBranch{Discrete Models and Algorithms}

% An optional dedication: you can thank whomever you wish (your supervisor,
% consultant, a person who lent the software, etc.)
\def\Dedication{%
I hereby give my thanks to my supervisor, Jan Hubička, for his valuable input, helpful
advice and language corrections. In addition, I thank my family and friends,
who supported me during my studies.
}

% Abstract (recommended length around 80-200 words; this is not a copy of your thesis assignment!)
\def\Abstract{%
Both major open-source compilers, GCC and LLVM, have a mature link-time
optimization framework usable on most current programs. They are however not
free from many performance issues, which prevent them to perform certain
analyses and optimizations. We analyze bottlenecks and identify multiple places
for improvement, focusing on improving interprocedural points-to analysis. For
this purpose, we design a new data structure derived from Bloom filters and use
it to significantly improve performance and memory consumption of link-time
optimization.
}
\def\CzAbstract{%
Oba vedoucí open-source překladače, GCC a LLVM, mají vyspělé optimizéry celých
programů, použitelné pro většinu současného softwaru. Stále však trpní mnoha
problémy s výkonem, což zapřičiňuje nemožnost použít některé analýzy a
optimalizace. V této práci analyzujeme problémová místa a identifikujeme
několik kandidátů na vylepšení. Pro tento účel vyvineme novou datovou struktur
založenou na Bloomových filtrech, díky které docílíme výrazného zlepšení časové
i paměťové náročnosti během optimalizace celých programů.
}

% 3 to 5 keywords (recommended), each enclosed in curly braces
\def\Keywords{%
{compiler}, {link-time optimization}, {points-to analysis}, {data structures}
}
\def\CzKeywords{%
{překladač}, {optimalizace celých programů}, {points-to analýza}, {datové struktury}
}

%% The hyperref package for clickable links in PDF and also for storing
%% metadata to PDF (including the table of contents).
\usepackage[pdftex,unicode]{hyperref}   % Must follow all other packages
\hypersetup{breaklinks=false}
\hypersetup{pdftitle={\ThesisTitle}}
\hypersetup{pdfauthor={\ThesisAuthor}}
\hypersetup{pdfkeywords=\Keywords}
\hypersetup{urlcolor=blue}

% Definitions of macros (see description inside)
\include{macros}
\usepackage{kmath}

% Title page and various mandatory informational pages
\begin{document}
\include{title}

%%% A page with automatically generated table of contents of the master thesis

\tableofcontents

%%% Each chapter is kept in a separate file
\chapter*{Preface}
\addcontentsline{toc}{chapter}{Preface}

%Dosavadní implementace alias analýzy v GCC je založena na článcích [Efficient
%Field-sensitive pointer analysis for C" a "Ultra-fast Aliasing Analysis using
%CLA: A Million Lines of C Code in a Second".
%
%Pro výpočet analýzy používá constraint-graph, který obsahuje výrazy typu
%Dereference, vzetí adresy, a skalární, nad kterými spočítá (?) tranzitivní
%uzávěr a vydedukuje points-to množiny. 
%
%Toto má mnohé nevýhody, zejména praktickou nepoužitelnost při optimalizacích
%celých programů (? IPA a LTO).
%
%Právě na téma škálovatelnosti se v posledních letech upřely zraky mnohých 
%akademiků i programátorů. Příchod LTO a existence programů obrovských rozměrů, 
%jako je například Firefox či KDE, zapříčinily, že standardní postupy již nejsou 
%použitelné.
%
%Cílem této práce je navrhnout algoritmy a datové struktury, které budou
%dostatečně rychlé a prostorově úsporné, aby byly použitelné při kompilaci
%velkých programů bez potřeby superpočítačů.

As soon as programs started their growth, it became necessity to split them into
functions, and later into compilation units. This shields the programmer from
unnecessary technical details of implementation, and allows him to concentrate on
the actual work.

The same applies on the compiler, which is shielded from the implementation
details. In this case it is a major disadvantage, as the compiler needs as much
information as it can get, to generate better code.

For a long time, most compilers worked on separate compilation units, and didn't
really care about other units in terms of analysis.  In the recent years
compilers grew more capable and computing power of consumer-grade machines
increased to the point where advanced optimizations could be performed even on
very large programs.

In 2009, the GCC project merged link-time optimizer into version 4.5, which enables
analysis and optimization on scope of all compilation units. This offers new
opportunities for improvement, but also new challenges. Some of the existing
algorithms can easily process whole programs, some have limitations, and some
of them are just too slow and/or memory intesive to be used in production.

The goal is to explore current link-time optimization techniques, identify a
bottleneck, and improve upon it.

We will introduce compilation and link-time optimization techniques in the
first chapter, introduce Alias Analysis problem and it's possible solutions in
the second chapter, with emphasis on practical use in current compilers. We
will improve upon Andersen's inclusion-based algorithm in the third chapter, by
using efficient data-structure derived from Bloom filters, sacrificing some
precision for tractability, and compare the results with non-approximate
solution using the same algorithm. In the last chapter we will provide
documentation necessary to reproduce presented results.



%Klasický přístup k překládání programů je založen na separaci kódu tak, že
%jednotlivé překladové jednotky (typicky složené z několika málo souborů) se
%přeloží najednou, a až po přeložení všech jednotek se program slinkuje do
%jednoho výsledného objektu.
%
%Tento přístup šetří zdroje počítače, který překlad provádí: jednak není nikdy
%zapotřebí mít nahraný celý program v paměti, dále je možno paralelizovat a
%překládat jednotlivé jednotky nezávisle. Nadruhou stranu však tento přístup
%zbraňuje překladači provádět optimalizace na rozhraní překladových jednotek.
%
%S tím, jak roste síla počítačů, a zejména dostupné paměti, je možné kompilační
%jednotky dále zvětšovat a pokoušet se o optimalizaci programů jako celku.
%
%Některé projekty se daly vlastní cestou, a tento problém vyřešili
%předzpracováním, které celý zdrojový kód vložilo do jednoho souboru; příklad
%může být SQLite, oblíbená relační SQL databáze, v jednom souboru obsahujícím cca
%370 řádek kódu v jazyce C.
%
%U větších projektů, jako například KDE, známé desktopové prostředí, se však brzy
%stane tento postup neunesitelný, a to zejména díky zdrojům, které jsou zapotřebí
%na překlad programu, a také díky nemožnosti snadno paralelizovat, což v dnešní
%době vícejádrových procesorů je nedostatek zcela zásadní.
%
%Tato bariéra je částečně odstraněna zavedení možnosti vložit do částečně
%přeloženého objektu dostatek metadat na to, aby ve fázi linkování šlo spustit
%některé optimalizační průchody, tzv. LTO (Link Time Optimization).
%
%Některé optimalizační průchody však byly navrženy dosti neefektivně s tím, že
%fungují dobře na jednotlivých funkcích, uspokojivě na menších překladových
%jednotkách, ale již dobře neškálují na celé programy, a to jak potřebným
%výpočetním časem, tak prostorem, a to i o několik řádů.
%
%Cílem této práce je zanalyzovat stav optimalizací, které aktuálně nejsou
%použitelné, navrhnout a implementovat řešení.
%

\newcommand{\definice}{\paragraph{Definice.}}

\chapter{Compilation and optimization}

In the beginning of this chapter, we will look into the composition of modern
programs, their codebase size and organization. We will continue with an
overview of compiler's workflow, introduce optimization passses and link-time
optimization framework.

\section{Code organization and codebase size}

Let us start by examining some of the code bases for programs we use every day.
A lot of developers run Linux, Firefox (or other browser) and GCC every day, but
unless we're developing one of them, we don't really have a good idea of how large
they are. It is not hard to guess they are enormous, containing millions of
lines of code. But how many exactly, and how is the number growing?
The chart in Figure \ref{figure-loc} show historical development over the past
10 years.

\begin{figure}[h!]
\label{figure-loc}
\centering
	\hspace{-1cm}\includegraphics{graphs/loc/loc.pdf}
\caption{Codebase size of Firefox, Chrome and GCC over time. [Data provided by
	openhub.net]}
\end{figure}

It might be tempting to say that the code will be split into many libraries. It
is no secret that for example Firefox bundles many libraries inside, they might
be compiled separately, resulting in many self-contained libraries in the
distribution. Figure \ref{figure-firefox-objsize} shows 8 largest libraries
contained in a standard Firefox distribution. The main library is 66.39 MB
large, the second largest is only 1.51 MB. This is due to speed optimizations,
as the developers noticed a significant start-up delay if all libraries are
loaded separately, and bundled them together into a single library. This also
means that compiler optimizing this library has to deal with all the code at
included.

\begin{figure}[h!]
\label{figure-firefox-objsize}
\centering
\includegraphics[angle=-90,trim=0 30 10 0,clip]{graphs/firefox-objsize/objsize.pdf}
\caption{Firefox 50.0.2 object sizes by binary}
\end{figure}

We should keep these numbers in mind while writing a compiler.

\section{Program compilation}

Only the simplest programs consist of a single source file, many programs have
tens, hundreds or even thousands of source files. This not only serves an
organizational purpose, but also allows the programmer to choose different
optimization flags for different files, or even write different parts in
different languages. A mechanism called {\sl separate compilation} is used to
compile and combine (link) all of them together, to form a finished program.
Figure \ref{figure-non-lto-workflow} shows the transformation using standard GCC
and Binutils (compiler and linker).

First step is to compile every  source file into object file. In this phase, a
compiler is invoked and does all the work necessary to convert source code into
binary, including code generation. The result is stored in an object file,
including required metadata, for example symbol table. This step is independedt
for each source file, so they can be processed in parallel. 

Second step consists of simple linking. A linker inspects all generated objects,
resolves required and provided symbols, dynamic libraris, and produces a
final executable. The linker usually doesn't modify code in object files, as it
only understands symbols and sections.

\begin{figure}[h!]
\label{figure-non-lto-workflow}
\centering
\includegraphics{./img/non-lto-workflow.pdf}
\caption{Standard workflow for separate compilation}
\end{figure}


\section{Compilation phases}

To compile a program written in high-level programming language into a machine
code requires many steps. To make orientation easy, and to support code
re-usability, a compiler usually consists of these three parts:

\paragraph{Front End} understands the input language, builts abstract syntax
tree and converts in into a common intermedialy language (IL) to all front ends
and the middle end.

\paragraph{Middle End} analyses the IL code and does most highlevel
optimizations. This includes splitting the code into basic blocks, building a
Control Flow Graph (CFG), dead code elimination, constant propagation, profile
driven transformations and many others \TODO{Ref forward.}

\paragraph{Back End} converts IL code into machine code, optimizing on the
lowest level, being able to schedule individual instructions and registers.

During this process multiple intermediary languages are used, sometimes more at
the same time (usually during transition to the lower-level language).

\begin{description}
	\item[GENERIC] is the highest-level IL used by the Front End, able to
		represent syntax trees and language-specific features.
	\item[GIMPLE] is tuple-based IL language, able to represent only simple
		expressions common to all languages. It is unable to represent many
		high-level constructs as for example loops.
	\item[RTL] (Register Transfer Language) is a low level language similar to
		machine code, containing algebraicly described instructions, as should
		be generated.
\end{description}


\section{Link-time optimization}

The idea of interprocedural and link-time optimization (LTO) is old. One of the
first ideas were published in 1970s \cite{Allen1974}, \cite{Allen1976}. It was
already supported in MIPSPro in 1990s\TODO{CITE}, which was later released in 2000 as
Open64 under GNU GPL. The compiler suite LLVM supports link-time optimization by
design, from their first release in 2002 \cite{lattner2002llvm}.  GCC was late
with their link-time optimization framework was proposed in 2005 [Ref:DraftLTO],
[REF:DraftWHOPR] and released in 2009.

Programs are usually compiled separately, meaning every source file is compiled
into it's own object file. All the object files are then linked together,
forming the final binary. This is a good technique, as it not only allows
logical grouping of code into files, but also change in one file does not force
recompilation of all other files. On the other hand, it also means that
optimizer only sees one file at a time, and some optimizations might be hard or
even impossible to do.  One example might be devirtualization in C++. As the
class it's descendants are usually in a separate file, the compiler has no way
of knowing that there is only one possible virtual method, and devirtualize it.
This often happens when using Mock objects\footnote{Objects that mimic some
behavior, for example a piece of hardware} for testing.

Some authors worked around this limitation. For example SQLite or older versions
of KDE support code concatenation in their build system. This results in one
huge source file being passed to the compiler. The result was good in it's day,
but still has some issues. All the code needs to be parsed at once, which
increases memory usage and does not scale well, as language parsers are not
usually parallel, and thus cannot make use of multi processor system.

\begin{figure}[h!]
	\label{figure-lto-workflow}
	\centering
	\includegraphics{./img/lto-workflow.pdf}
	\caption{Compiling source code into binary}
\end{figure}

The LTO framework (see Figure \ref{figure-lto-workflow}) solves these problems by keeping separate compilation, but
instead of generating classical object files containing machine code, the
middle end stops and writes GIMPLE representation into the object, including
some metadata (for example the call graph).

Instead of generating library, the linker then picks up the GIMPLE
representation and invokes the compiler again. GCC was designed to perform most
of the optimizations in parallel, and the process has been further split into
two stages. The sequential WPA (Whole Program Analysis) stage and parallel
LTRANS (Local TRANSformations) stage.

\paragraph{WPA} stage performs declaration and type linking, and decision stage
of interprodecural optimizations. It ends by partitioning the code into smaller
pieces called {\sl LTO partitions}. The partitioning happens with regard to the
code being optimized, for example to minimize cross-partition edges.

\paragraph{LTRANS} stage then performs optimizations decided by WPA stage,
followed by local optimizations and code generation.


\section{Evaluating GCC performance}

As we previously noted, we will be focusing on the GCC compiler suite. In this
section, we will cover the experimental setup, evaluate compile time in a few
programs and extract practical information on resources needed and
anticipated use.

We will publish source code used to take these measurements, so anyone can
reproduce these results, and perhaps use for comparison on his/her own work.We 
will also omit some technical details in this chapter, but will include them in
the Appendix \TODO{link}.

\subsection{Compiler}

For further measurements, we will be using GCC compiled from branch {\tt
gcc-5-branch} (via github.com mirror, but any up to date repository will
suffice). The reason for this branch is simple: it's relatively fresh branch,
that supports most of the latest features, but will not change during
development and provide stable base for testing, while still receiving bug fixes
for the time being. Most of the benchmarking is non-intrusive but some involve
large data dumps which may slow us down. This will be noted whenever appliccable.


\subsection{Software under test}

Many opensource programs are available for testing, however a good candidate has
to fulfill a few requirements to make testing straightforward and reproducible.

\begin{itemize}
	\item Written in C++ (preferred) or C.
	\item A good compatibility with current GCC versions (5.x and 6.x)
	\item Flexible and robust build system.
	\item Mid to large codebase.
	\item Preferably monolithic binary.
\end{itemize}

It is surprisingly difficult to find projects that fit all of those
requirements. Many popular programs were ruled out immediately. For example the
build systems of MySQL and Inkscape is very inflexible and has trouble with LTO
compilation. GIMP consists of many plugins which doesn't pose any challenge for
the current link-time optimizer. However, a few others passed:

\paragraph{Firefox} was an easy choice, as it is an established benchmark for GCC
and though there were a few issues in the beginning, later versions are polished
and fullfill all the requirements. There is only one problem, and that is it
takes approximately 25-36 hours to compile firefox with non-modified GCC
sources, which isn't very good for development. The main reason for this
slowdown is excessive RAM usage, which doesn't allow parallel LTO pass.  
The specific version used in testing is {\it Firefox 48}.

\paragraph{Merkaartor} is an OpenStreetMap editor written in C++ with medium
sized code-base. It utilizes Qt framework, a lot of C++ constructs and links
plenty of objects into a single binary, and uses a lot of C++ constructs.

\paragraph{SQLite} is a SQL database engine in a single source file with a
medium (to small) sized code-base. It offers fast compilation times 

The specific version used in testing is {\it Merkaartor 0.18.3-rc1}

\subsection{Other software and hardware}

\subsubsection{Hardware}

Where relevant, the following machine was used for testing:

\begin{itemize}
	\item Intel Xeon E3-1231-v3 @ 3.40GHz (Haswell)
	\item 32GB DDR3 RAM @ 1600MHz, consisting of 4 modules KHX1600C10D3/8G
	\item 120GB Intel 520 SSD, SSDSC2CW120A3
\end{itemize}

I consider this setup to be high-end for desktop computing, and much more than
should be required. Other machines were used during testing, but all time
results will be quoted for the one above.

\subsubsection{Software}

The system was running 64bit Linux kernel 4.5 and standard Gentoo Linux
installation. 

Memory and cpu usage measurements were taken using Linux Control Groups for
whole compilation process, including GNU make and other tools. The data were
sampled at 1 second intervals, which is more than enough. Total CPU usage is
known precisely, as control groups keep cumulative counter. The activity at a
given point is used only as a pointer as to how many cores are currently
computing. As for the memory, the second interval is fine as well, as we are not
allocating and freeing memory rapidly. In fact, most of our allocations will be
done at the beginning of an analysis.

\begin{figure}[h!]
	\label{figure-firefox-ipa-kpta}
	\centering
	\includegraphics{./graphs/firefox-ipa-kpta/firefox-ipa-kpta.pdf}
	\caption{Building libxul.so with {\tt -fipa-kpta -flto=8}}
\end{figure}

\chapter{Alias Analysis}

The goal of Alias Analysis is to determine the ways a memory location can be
accessed. In the language C, a location is usually accessed by its name or a
pointer. Disambiguating two accesses is necessary for many optimizations, but
precision is also needed for the optimizations to be correct. See example in
Figure \ref{alias-example1} which demonstrates both the cases. The second
assignment to {\tt b} might seem redundant, as {\tt a} could not have changed.
However, it is true only if the call to {\tt some\_fn} did not change variable
{\tt b}. 

\begin{figure}[h!]
	\label{alias-example1}
\begin{tcolorbox}
\begin{lstlisting}{language=c,tabsize=2}
void set_call_set(void) {
	int a,b;
	[...]
	b = a + 1;
	some_fn(a, &b);
	b = a + 1;
	[...]
}
\end{lstlisting}
\end{tcolorbox}
	\caption{Example of the importance of alias information}
\end{figure}

In a large program the aliasing may be arbitrarily complex, but many
optimizations will be possible even with a minimal aliasing information.
Consider example in Figure \ref{alias-example2}. The loop seems to write zeroes
into an array of floats {\tt a}. This is true if the pointer dereference can not
change the pointer itself. Fortunately we do not need to examine any code not
shown in the example. The C standard prohibits the dereference of {\tt float*}
to modify {\tt float*} itself.

\begin{figure}[h!]
	\label{alias-example2}
\begin{tcolorbox}
\begin{lstlisting}{language=c,tabsize=2}
void fill_floats(void) {
	float* a;
	[...]
	for (int i = 0; i < 10; i++)
		*(a++) = 0;
}
\end{lstlisting}
\end{tcolorbox}
	\caption{Example of the importance of alias information}
\end{figure}

To answer questions about aliasing, we usually use {\it Alias oracle}.
This usually is a function that given two memory accesses in a program
answers if they can access the same memory location. The answer can be {\it
yes}, {\it no} or {\it maybe}. A single oracle can apply multiple algorithms to
determine the answer. The following three oracles are most often used:

\subsection{Type based analysis}

As the name suggests, Type Based Alias Analysis (TBAA) infers aliasing
information from types and language rules, as seen in Figure
\ref{alias-example2}. This method is very fast, as it only needs to inspect
the types in question. Due to the speed, it is usually asked first and is able
to distinguish many cases by itself.

\subsection{Base and offset analysis}

If two accesses possibly refer to the same language type, they may not
necessarily refer to the same memory location if the object is compound, for
example a C {\tt struct}. The base structure and offset in it can be compared.
For example, if a {\tt struct} contains two integers, access to one of them is
distingueshed from the other by base and offset analysis.

\subsection{Points-to analysis}

If a memory access cannot be disambiguated by any simpler rule, points-to sets
must be analyzes and evaluated. A {\it points-to set} for a variable is a set of
memory locations the variable can be used to access. For example, a simple
non-pointer variable can only be used to access itself (access by name). A
pointer variable can be used to access other memory locations of which the
address was taken. To disambiguate memory accesses the points-to sets have to be
compared and if their intersection is empty, it is safe to assume they do not
alias. If their intersection is non-empty, or some of the sets could not be
computed, we must assume they do intersect to preserve correctness.

Compared to type based and base and offset analysis, points-to analysis is a
time-consuming process and will be a focus of this chapter.

\section{May and Must-alias}

It is useful to distinguish between may-alias and must-alias information.
May-alias information indicates that the access may alias on at least one path
in the control flow graph. On the other hand, must-alias information requires
alias in all possible paths. Consider example in Figure
\ref{alias-example-maymust}. The information that ``{\tt p} points-to {\tt x} or
'{\tt y}' is an example of may-alias, as it depends on the condition taken.
The information ``{\tt q} points-to {\tt x}'' is an example of must-alias, as it
holds on all paths in the example. Notice that must-alias always returns a
single element, may-alias usually returns larger points-to sets.

\begin{figure}[h!]
\label{alias-example-maymust}
\begin{tcolorbox}
\begin{lstlisting}{language=c++,tabsize=2}
void fill_floats(void) {
	int *p,*q;
	int x,y;
	[...]
	q = &x;
	if (x > y) {
		p = &x;
		[...]
	} else {
		p = &y;
		[...]
	}
	*p = 0;
	[...]
	
}
\end{lstlisting}
\end{tcolorbox}
\caption{Example of may and must-aliasing}
\end{figure}

\section{Flow and context sensitivity}

It is also useful to distinguish between flow sensitivity, context sensitivity
and their combinatins.

A {\it flow-sensitive} algorithm computes the alias information with regard to control
flow. In the example in Figure \ref{alias-example-maymust} it would notice the
different branches of {\tt if} and provide information that ``{\tt p} points to
{\tt x} in the {\tt if} branch'' and similarly for the {\tt else} branch.

A {\it flow-insensitive} algorithm computes alias information without any regard
to control flow. In the same example it would just output ``{\tt p} may point-to
{\tt x} or {\tt y}''.

Also notice that the flow-sensitive algorithm gave us must-alias
information, while the flow-insensitive only gave may-alias information.
This is to be expected, as the flow-insensitive variation would require only
single assignment to the pointer (at initialization) to output must-alias
information.

Context sensitivity is a similar problem to flow sensitivity but in
intraprocedural case. While flow sensitivity relies on control flow graph inside
a single function, context sensitivity is based on callgraph. 

Before we continue further, let us formally define the various alias-analysis
types. We will use definitions compatibile with  \cite{muchnick1997advanced}.

\paragraph{Definition} {\it Flow-insensitive may-alias information} is a binary
relation $A_{FinMay} \subseteq \Var \times \Var $ on the variables. A pair
$(x,y)$ is in the relation if and only if $x$ and $y$ can refer to the same
memory location, possible at a different place in the program, or at a different
time during execution. This relation is symmetric, but is not transitive.

\paragraph{Definition} {\it Flow-insensitive must-alias information} is a binary
relation $A_{FinMust} \subseteq \Var \times \Var$ on the variables. A pair
$(x,y)$ is in the relation if and only if $x$ and $y$ always refer to the same
memory location during the program execution. This relation is symmetrict, but
also transitive. 

The flow-sensitive case is a bit more complicated, and can be examined both as a
relation or function.

\paragraph{Definition} {\it Flow-sensitive may-alias information} is a ternary
relation $A_{FseMay} \subseteq \Var \times \Var \times \Loc$ on the variables
and program locations. A triplet $(x,y,p)$ is in the relation if and only if $x$
and $y$ can refer to the same memory location at the point $p$ in program
execution.

\paragraph{Definition} {\it Flow-sensitive must-alias information} is a ternary
relation $A_{FseMust} \subseteq \Var \times \Var \times \Loc$ on the variables
and program locations. A triplet $(x,y,p)$ is in the relation if and only if $x$
and $y$ always refer to the same memory location at the point $p$ in program,
regardless of what the memory location is.

A similar definition could be used for context sensitivity, adding call context
to the relation as well, or encoding it in the location. The specifics depend on
the definition of context, as there are multiple possiblites. A context could be
just a call site, or a path in callgraph from the entry point, possibly only to
a limited depth.

%\section{Problem complexity}
%
%Consider the  C code in Figure \ref{figure-alias-evil}, where {\tt X}, {\tt Y}
%and {\tt Z} are some constants and {\tt x29A} is some unknown function. Let us
%analyze what the points-to set for {\tt p} could be. It is not clear what the
%points-to set for {\tt p} should be. During execution, any of the following
%three cases could arise:
%
%\begin{enumerate}
%	\item Function {\tt x29A} never returns, {\tt p} points only to {\tt NULL}.
%	\item Function {\tt x29A} finishes and always returns true or always returns
%		false, {\tt p} points to either {\tt *X} or {\tt *Y}.
%	\item Function {\tt x29A} finishes and sometimes returns true, sometimes
%		false, {\tt } points-to {\tt *X, *Y}.
%\end{enumerate}
%
%Fortunately the case 1. does not need to be solved, as the {\tt p} would never
%be used. Even constructing infinite function in most languages is hard. For
%example in C, this can be done by three things: program exit, infinite recursion
%or infinite loop. Program exit causes {\tt p} not to be used and we can disregard this case. 
%As computers have limited amount of RAM, infinite recursion is not possible, so
%we can disregard this case as well. The only one left is infinite loop, which is
%possible, but in {\tt C}, the standard permits to assume most loops will
%terminate\cite{isoc} and even if the they do not, we can assume they do, as the
%code will not not execute if they do not. This is good news, as we do not have
%to solve halting problem during alias analysis.
%
%If we had to recognize and analyze the case 2., the alias analysis would have to
%decide the answer of {\tt x29A}. As the inputs are constants, it theoretically
%could be done, but the author could also encode an NP-hard problem in them, and
%force our analysis to be arbitrarily complex.
%
%\begin{figure}[h!]
%\label{figure-alias-evil}
%\begin{tcolorbox}
%\begin{lstlisting}{language=c++,tabsize=2}
%bool x29A(void **arg1, void *arg2, char *str);
%
%int main() {
%	void *a = X;
%	void *b = Y;
%	void *p = NULL;
%	p = x29A(a, b, Z) ? a : b;
%	[...]
%}
%\end{lstlisting}
%\end{tcolorbox}
%\caption{Example of AA complexity}
%\end{figure}

\section{Problem complexity}

It is useful to know how difficult the problem of alias analysis is. In this
section we will review previous results showing the theoretical bounds for
different problem variants.

The earliest classification is from Landi \cite{Landi1991}, who proves that
computing flow-sensitive may- and must-alias information in the presence of
single level pointers can be done in polynomial time. By adding more levels of
indirection, as is common in most languages, the problem becomes NP-hard.

Later Horowitz \cite{Horowitz1997}  proved that precise flow-insensitive alias
analysis is NP-hard with only scalar variables and no heap allocations, though
the result assumes unrestricted pointer dereference.

Chakaravarthy \cite{ptcomp} proved that when heap allocations are allowed the
problem becomes undecidable, even if all the variables are scalar. The same
articles also proves that the flow-insensitive variant is in $\P$, if the
variables are further restricted to well-defined types\footnote{Known type and
number of indirections}. Although this is not always the case, it gives us hope
that a successful alias analysis could be performed on a well-formed program.


%\subsection{Single-level pointers}
%
%Let us first analyze the complexity of may-alias in a simple case, where only
%single level pointers exist and no dynamic allocation is possible. Denote by $n$
%the number of pointers, by $v$ number of scalar variables and memory locations.
%
%To compute points-to sets, we start with empty points-to sets, and insert
%variable in each instance its address was taken. That is for each statement {\tt
%p := \&a}, we insert {\tt a} into points-to set for {\tt p}. The next step is to
%propagate across assignments between pointers, until we reach a stable solution.
%
%\begin{enumerate}
%	\item For each pointer variable $p_i$, let $PT := \{ 0_i \}$, where $0_i$ is
%		the initial value of $p_i$.
%	\item Propagate points-to sets for each modification of $p_i$.
%\label{triv-alg-prop}
%	\item If any set was changed in step \ref{triv-alg-prop}, go back to
%		\ref{triv-alg-prop}.
%\end{enumerate}
%
%We can easily formulate this problem as system of set inequalities where:
%
%\begin{itemize}
%	\item $p_i := \&a$ is $a \in p_i$ for $a$ scalar
%	\item $p_i := p_j$ is $p_j \subseteq p_i$
%\end{itemize}
%
%As there are only single level pointers, we are not allowed to take address of a
%pointer, and dereference always results in a scalar, which does not change any
%pointer.
%
%Now it's clear that when the algorithm exits, the solution is conservatively
%correct. Also, if $n$ and $v$ are finite, it will finish after at most $n \cdot
%v$ steps, as in each step at least one set will increase in size, and every set
%can have at most $v$ elements in.
%This gives us simple (not necessarily efficient) polynomial-time algorithm.
%
%Of course there is a problem that in practice we  do not have all the information.
%That may include external functions (possibly with side-effects), dynamically
%allocated memory and more obscure, possibly language-specific, features as for
%example pointer arithmetic. We will ignore these now for simplicity, and
%deal with them later. \TODO{Reference.}
%
%\subsubsection{Multi-level pointers}
%
%Things start to be difficult when multi-level pointers come in play. 
%\TODO{Proof.}

\section{Known algorithms and approaches}

During the years, only a few algorithms have been developed and as alias
analysis is a typical dataflow problem, there is little reason to expect a
practical but fundamentally different algorithm.

\subsection{Andersen's algorithm}

First published by Lars Ole Andersen \cite{Andersen94}, it is an {\it
inclusion-based} algorithm is based on direct mathematical representation of
aliases as points-to sets. That is, a points-to set for a given pointer $p$ is a
set $S_p$ containing all locations pointer $p$ can point to.  Further
expressions are then translated into set inequalities:

\begin{align}
	\label{aa-init}
	p_i = \&a \quad &\to \quad a \in p_i \\
	\label{aa-prop}
	p_i = p_j \quad &\to \quad p_j \subseteq p_i \\
	\label{aa-deref}
	p_i = *p_j \quad &\to \quad \forall p_k \in p_j : p_k \subseteq p_i
\end{align}

The structure of proposed Andersen's flow-insensitive algorithm is shown in Figure
\ref{figure-andersen}.

\begin{figure}[h!]
\label{figure-andersen}
\begin{tcolorbox}
\begin{enumerate}
	\item Initialize variables using \ref{aa-init}.
	\item Build a propagation graph using \label{aa-prop} and \label{aa-deref}
		with variables as vertices, propagations along edges.
	\item Find strongly connected components in the grah and merge them into a single node.
	\item Mark every node as changed.
	\label{aa-propstep} 
	\item For every changed node, reset its changed status, propagate the change
		along edges and mark nodes as changed if they were modified.
	\item Repeat step \ref{aa-propstep}. until no node is marked as changed.
\end{enumerate}
\end{tcolorbox}
\caption{Andersen's algorithm}
\end{figure}

\subsection{Steensgaard's algorithm}

Developed by Bjarne Steensgaard \cite{Steensgaard96}, it is  a {\it
unification-based}, similar to Andersen's, but uses unification instead of
subset contraints. It only needs to partition pointers into equavalence
classes, which can be done in almost linear time. This leads to a very fast and
scalable algorithm, sacrificing some precision.

There is very little research utilizing unification-based algorithms, as it is
believed to be patented by Microsoft\cite{patent:steensgaard}. It was
implemented in LLVM, but later removed in 2006 \cite{LLVM:DSA:Remove} due to
patent concerns. We expect this to change, as the patent has just expired while
writing this thesis, in September 2016.

\subsection{Further improvements}

Steensgaard's algorithm can use Union-Find data structure for the unification,
which is already extremely efficient. Andersen's algorithm has to deal with
sets, and the choice of data structure for set management is harder. Two major
improvements have been proposed to date, though none of them have been
implemented in a production compiler.

\subsection{Bloom filters}

The use of Bloom filters was first proposed by Nasre et. al \cite{nasre2009}.
They are very space efficient and perform well on certain operations, as is
query and union. Some implementations can also perform interesection, but with
decresed precision. The complete lack of the ability to enumerate elements was
worked around by introducting multiple dimensions for multi-level pointers. In
this scheme, a pointer could be easily dereferenced upto a constant depth and
after that, the algorithm answers conservatively.

We will revisit the use of Bloom filters in later chapters.

\subsection{Binary decision diagrams based algorithms}

A Binary decision diagram (BDD) is a data structure used to represent boolean
functions. It can be easily extended to represet relations by encoding
characteristic function of given relation, and the complete alias information as
well. Multiple algorithms based on the BDDs were developed \cite{whaley2004},
\cite{bddbddb}, but most of them lack public and usable code for further
development. The major issue with the use of BDDs is that they heavily rely on
the correct variable ordering. Choosing wrong ordering quickly results in
size explosion and speed decrease. However the BDD approach seems promising for
loss-less representations.

\section{Current state in compilers}

There are not many modern compilers with open code that can be examined and improved
upon. One of the players is GCC, that has been around since 1985
(1.x release was in 1991) and is the most widely used open source compiler
today. The younger competitor is LLVM/Clang, first released in 2003. It's
written in C++, is supported by Apple since 2005, and due to it's age has much
mure modern design, and is generally deemed to be easier to extend and work
with. A lot of researchers also focus on Java compiles and algorithms, and though many
techniques can be used for C and C++, Java is very different language, in that
it has JIT\footnote{Just In Time} compiler, and does not have pointers in the
classic sense, only references, which simplifies some cases.

There are much more compilers available, but most of them are proprietary or not
maintained, as for example the Open64, ICC and VisualC++.

Also, it is very hard to compare many of the published results, as the
implementations are not public, and mostly implemented for compilers that are
unable to keep up with current C/C++ standards and successfully build modern
(and big) projects. Many of the results are computed outside the compiler and
never tested. But even if they were, there is no simple metric that could be
used for comparison. The results rely on previous optimization passes,
constraint generator, chosen granularity (wether to consider structure members or
arrays) and finally queries asked by the compiler in later optimization phases.


\subsubsection{LLVM/Clang}

As LLVM is very modular, it contains multiple alias analysis passes. 

\begin{itemize}
	\item {\bf -basic-aa} pass, providing local alias information using many
		language-specific facts.
	\item {\bf -scev-aa} pass, translation queries into Scalar Evolution queries
		\TODO{WTF?}
\end{itemize}

Additionally, there are three passes in {\tt poolalloc} package:

\begin{itemize}
	\item {\bf -globalsmodref-aa} pass, providing context-sensitive alias
		information for global variables
	\item {\bf -steens-aa} pass, implementing Steensgaard's algorithm.
	\item {\bf -ds-aa} pass, implementing unification-based Data Structure
		analysis, providing context and field sensitive alias information.
\end{itemize}

\TODO{Are they interprocedural? steens probably is, but needs checking; more
info on this here: http://llvm.org/docs/AliasAnalysis.html}

\chapter{From Bloom filters to Bloomaps}

During points-to analysis the datastructure is almost as important as the
algorithm used. In the case of GCC, the structure chosen is a hybrid of bitmap
and a linked list. This works fine for small and dense data, but not so well
with large data.

Converting to a better data structure is relatively easy task, but what data
structures are available? We will start by examining the needs of a typical
Andersen-style algorithm and comparing theoretical complexities of various well
known data structures. In the rest of this chapter we will describe a new
enhancement of bloom filters called Bloomaps, tailored specifically for the use
in points-to analysis.

\section{Requirements}

We need a data structure holding sets of integers that is compact and has the
following operations: {\tt INSERT}, {\tt QUERY}, {\tt INTERSECT} and {\tt
ENUMERATE} and {\tt UNION}; each of them with a specific purpose:

\begin{itemize}
	\item The {\tt INSERT} operation is called only at the initialization.
	\item The {\tt QUERY} operation is mostly called for a few special values,
		and they can be handled separately, or at the end.
	\item the {\tt INTERSECT} operation is called at the end, to answer alias
		oracle queries.
	\item The {\tt UNION} operation is called many times in each iteration, for
		every dependency on a changed set.
	\item The {\tt ENUMERATE} operation is called only on pointer dereference.
\end{itemize}

In other words, the basic operations can be a bit slower, but we need a good and
fast {\tt UNION}. There are a few other considerations:

\begin{itemize}
	\item The {\tt UNION} will be called on the same pairs over and over again,
		which means most of the elements will be already in both.
	\item The average number of stored elements will be small, and the data sparse.
	\item Some sets may grow very large, containing almost every
		element possible.
	\item Low memory overhead is required, as the number of sets is in the order
		of number of elements inserted.
\end{itemize}

A simple bitmap is a straightforward solution, but the parseness make it
unviable. Many tree-like structures will have problems with the intersections
and union as all the elements have to examined and deduplicated.

A natural choice is a bloom filter, which is compact and has a fast union. The
compactness is a real treat, as we can go as far as to choose how much memory to
invest and sacrifice the precision if we don't have enough. Unfortunately, its
performance in {\tt INTERSECT} is rather suboptimal, and even worse, completely
lacks the {\tt ENUMERATE} operation.


\section{Bloom filters}

A bloom filter is a classical probabilistic data structure, invented by Burton
Howard Bloom \cite{Bloom1970}. The goal is to provide a data structure
that has some nonzero probability of {\it false-positive}, but zero probability
of{\it false-negative}. This is accomplished by taking a bit field of $m$ bits,
$k$ hash functions, and hashing every element into $k$ different bits, writing
$1$ on insertion, and checking if every position contains $1$ on query.

The Bloom filter has immediate applications in some areas, for example caching:
it is a good idea to ask a filter if an element is in the cache. If the answer is
no, we need to get it elsewhere. If the answer is yes, we can look into the
cache, and in the worst case it is not there (an ocurrence false-positive).

Here is a short list of properties (provided the hashes can be computed in
constant time, which is often possible).:

\begin{itemize}
	\item {\tt QUERY} in $\O(1)$ time.
	\item {\tt INSERT} in $\O(1)$ time.
	\item {\tt DELETE, ENUMERATE, RESIZE} not supported\footnote{Though there
		are variations that support these oprations, only {\tt RESIZE} is usually
		possible without drastic changes to the structure.}.
	\item {\tt UNION} in $\O(m)$ time (bitwise OR).
	\item {\tt INTERSECTION} in $\O(m)$ time (bitwise AND).
\end{itemize}

To be honest, the last one is bit of a lie, as it is not an intersection as one
would expect. Let's denote $BF(A)$ as a bloom filter created from empty filter
by inserting elements from $A$ one by one. Then:

For $A,B$ it does not hold that $BF(A \cap B) = BF(A) \cap
BF(B)$. Unfortunately, even the inequality $BF(A \cap B) \subseteq BF(A) \cap
BF(B)$ holds, it is nowhere near the equality. Most imporantly, we would like to
check for emptyness of intersection, which is hard to achieve.

\section{Bloom filter intersection}

Although a bloom filter intersection easily computed with bitwise {\tt AND},
it is rarely accurate. The basic properties are already understood:

As proven in \cite{bose2008false}, the probability that $BF(A\cap B) =
BF(A) \cap BF(B)$ is:
\begin{align}
p = (1-1/m)^{k^2\cdot |A-A\cap B| \cdot |B - A\cap B|}
\end{align}

For set emptyness, we can further simplify the formula by separating two cases:
$|A \cap B| > 0$ and $|A \cap B| = 0$. The second equality is interesting, as
we would like to test for empty sets.

Assuming that $|A\cap B| = 0$, let us compute the probability that $BF(A) \cap
BF(B)$ is also empty.
\begin{align}
	p_{empty} = (1-1/m)^{k^2 \cdot |A| \cdot |B|}
\end{align}

Furthermore, if we use partitioned Bloom filters, Jeffrey and Steffan
[REF:Jeffrey11] showed a slightly improved bound:
\begin{align}
	p_{empty} = \left(1 - \left( 1 - {k \over m}\right)^{|A|\cdot |B|}\right)^k
\end{align}

This is due to the fact that it is enough to have one empty partition to consider
the filter empty, as every query would result in false in the empty partition.

However, in the same article they proved that pure Bloom filter intersection is more
memory-consuming than more conventional queue-of-queries. In the next section,
a hybrid solution is provided that can be used with both of these approaches,
based on the time requirements.

\section{Bloom filter enumeration}

It's immediately clear that vanilla Bloom filter cannot provide list of it's
possible elements, as for example the simplest filter holding $1$ elements and
answering with false-positive probability $0.5$ would have to enumerate half the
universe $U$, which may as well be impossible for $U=\N$.

Besides the trivial queue-of-queries, there has been one attempt at constructing
Bloom filter-like structure, that can list it's items, the Invertible Bloom
Lookup Tables (IBLT), by Michael T. Goodrich \cite{goodrich:2011}. The problem of IBLT is
that they have non-zero probability of being unable to produce a complete list
of entries, and do not provide the advantages of classic Bloom filters, as a
fast intersection and membership queries. This said, we will not attempt to use
them, although they are an interesting structure for future work and may find
its use in other optimizers.

Let us review in short the available methods:

\paragraph{Enumerate entire universe.} This is possible for small and dense
universe, and appropriately sized filter. We don't have to keep extra data, but
queries need to ask for every element in a universe. Even for almost empty
filter, this approach takes $\O(|U|)$ time.

\paragraph{Keep a Queue-of-queries for each filter.} Perhaps the most sensible
way, if there are either few filters, or only a few elements in each filter.
This approach may fast insertion and query, though complexity of the
Queue-of-queries structure needs to be taken into account and either we need to
sacrifice more memory to store unsorted queue, or time to use a better data
structure for the queue.

Neither of these methods are good for our use, as our universe can be large
(millions of lines of code), and the number of filters is about the same size as
the universe. We introduce a new approach, a compromise between the two above.

\paragraph{Keep a single Queue-of-queries for all filteris.}

Assuming our universe is all 32 bit integers, we can store a bit array of every
item ever inserted under 512 MB. This allows the universe to be relatively
sparse (we can skip unused elements). Also the operations only need to perform
$\O(1)$ extra work to write a new bit, which is reasonable.

\section{Bloomaps and Families}

\paragraph{Definition.} Bloomap Family with parameters $(m, k, s)$ is a
datastructure that maintains a list of Bloomaps of the same parameters, and
indexed representation of used parts of the universe in union of all it's
Bloomaps, capable of enumeration.

\paragraph{Definition.} Bloomap is an enhanced Bloom filter, belonging to a single
family, capable of executing {\tt INSERT} and {\tt QUERY} itself, and {\tt
ENUMERATE}, {\tt UNION} and {\tt INTERSECT} within it's family.

A bloomap with parameters $(m, k, s)$ is constructed from a partitioned Bloom
filter with the addition of a {\it side index} containing $s$ bits. The side
index is used as another partition in the bloomfilter, however with simpler hash
function that is easily inverted (for example a simple {\tt SHIFT} and {\tt
AND} with a mask).

Furthermore, the Bloomaps and their families need to fulfill these conditions:

\begin{itemize}
	\item When a new item is inserted into a Bloomap, it is also inserted into
		the family.
	\item A Family has to enumerate all items inserted into it's Boomaps for any
		given hash.
\end{itemize}

\begin{figure}[h!]
\begin{tcolorbox}
	\begin{lstlisting}[language=c++,tabsize=2]
struct Bloomap {
    BloomapFamily f;
    int m,k,s;

    int index[s];
    int partitions[k][m/k];
};

struct BloomapFamily {
    vector Bloomap;
    int m,k,s;

    hash_set universe;
}
\end{lstlisting}
\end{tcolorbox}
\caption{Bloomap and BloomapFamily prototypes}
\end{figure}

Where {\tt hash\_set} is some data structure capable of storing a set of
elements from universe associated with a given hash. The naive C++ structure
might be {\tt hash\_map<vector<universe\_type>>}.

Before we get into technicalities, let us illustrate how {\tt INSERT} and {\tt
ENUMERATE} functions might be implemented:

\begin{figure}[h!]
\begin{tcolorbox}
\begin{lstlisting}[language=c++,tabsize=2]
void Bloomap::INSERT(element) {
    //Decompose element to offset, index_hash and data.
    (offset,index_hash,data) := element;
    //Insert into side_index of a bloomap.
    index[index_hash] := true;
    for (i := 1..k) {
        partitions[i][hash(i,element) % m/k] := 1
    }
    //Insert into universe index of a family
    f.universe[hash] += element;
}

void Bloomap::ENUMERATE(bloomap) {
    list = ();
	for (i := 1..s | index[i] == true) {
        for (element in f.universe[i] | element in bloomap) 
            list += element;
    }
    return list;
}
\end{lstlisting}
\end{tcolorbox}
\caption{Pseudocode for Bloomap::INSERT and Bloomap::ENUMERATE}
\end{figure}


Both of them are pretty straightforward, as {\tt INSERT} is a regular function
for Bloom filter insertion, with the added partition for side index and family
universe insertion.

We might expect this to run in $\O(1)$, as it does for Bloom filters. Writing to
the index doesn't make it worse, but inserting into a universe might. Depending
on the set implementation, we can expect additional $\O(\log n)$ for tree-based
implementations or amortized $\O(1)$ for hash-based implementations. We probably
can't do much better in generic case, but we will suggest a worst-case $\O(1)$
for 32 bit integers (dense sequence of ids starting from 0).


\subsection{Compact representation of dense integer universe}

Representing universe requires storing sets for different hashes. It's wasteful
to store them in a linked-list, trees or even hash tables, as a humble bit array
fulfills the task. A little unusual form of a bit array has been used, in order
to achieve less allocations and space efficiency.

As mentioned above, we will split the value to {\tt offset}, {\tt hash} and {\tt
data} at binary boundaries. This means we can simply concatenate the values to
get the represented integer. We can now organize the data into {\it buckets} and
{\it superbuckets} in the following way:

\begin{itemize}
	\item Each offset has it's own {\it superbucket}.
	\item Each {\it superbucket} contains a {\it bucket} for every {\tt hash} value.
	\item Each {\it bucket} contains a bit for every {\tt data} value.
\end{itemize}

\begin{center}
	{\tt element = offset$\cdot$hash$\cdot$data} $\Leftrightarrow$
	{\tt universe[offset$\cdot$hash].bit[data]}
\end{center}

\begin{wrapfigure}{r}{0.5\textwidth}
	\label{figure-bucketshop}
	\hfill
	\includegraphics{img/bucketshop.pdf}
	\caption{Bucket and superbuckets in an array}
	\vspace{1cm}
\end{wrapfigure}

The structure is illustrated in Figure \ref{figure-bucketshop} and pseudocode
implementation in Figure \ref{figure-bucketshop-pseudocode}. Memory allocation
is expected to be done automatically in {\tt vector} class and the array should
be resized on first access beyond current boundary.  This allows the structure
to occupy only as much memory as is necessary to store a set of size at most
$\O(\max(E))$.

\begin{figure}[h!]
	\label{figure-bucketshop-pseudocode}
\begin{tcolorbox}
\begin{lstlisting}[language=c++,tabsize=2]
struct superbucket {
	uint64_t bits[];
};

struct universe_index {
	vector<struct superbucket> superbuckets;
};

void universe_index::INSERT(offset, hash, data) {
	superbuckets[offset]->bits[hash].bit[data] = 1;
}

vector<element> universe_index::ENUMERATE(hash) {
	vector<element> candidates;
	for (sb in superbuckets) {
		for (i = 0 .. 63) {
			if (sb->bits[hash].bit[i]) 
				candidates.append(bit);
		}
	}
	return candidates;
}
\end{lstlisting}
\end{tcolorbox}
\caption{Bucket and superbuckets prototype and pseudocode}
\end{figure}




\chapter{Using Bloomaps in points-to analysis}

In this chapter we will discuss the internals of points-to analysis as
implemented in the GCC compiler and how it was augmented by the use of Bloomaps
instead of regular bitmaps. We will then show how the change affected GCC and
discuss future work.

\section{Points-to analysis in GCC}

The points-to analysis in GCC is implemented in two files. The {\tt
tree-ssa-alias.c} contains the alias oracle and the TBAA algorithm. Most of the
functionality is wrapped in function {\tt refs\_may\_alias\_p(tree,tree)}, but a
few others exist to check for aliasing with global memory, call clobbers and
other special cases.

The data is stored in a structure {\tt pt\_solution}, which is computed by
algorithm in {\tt tree-ssa-structalias.c}. It is an implementation based on
\cite{Pearce2004,Heintze2001}, and is an Andersen-style algorithm. An overview
of this algorithm is on Figure \ref{figure-gcc-aliasalg}.


\begin{figure}[h!]
\label{figure-gcc-aliasalg}
\begin{tcolorbox}
\begin{enumerate}
	\item Each variable get allocated a {\tt varinfo\_t} structure, which includes some metadata and a solution set $\Sol(x)$.
	\item All direct constraints ({\tt p = \&q}) are found and processed (used to initialize the solution sets).
	\item All copy constraints ({\tt p = q}) are found and used to build a constraint graph on variables, such that for {\tt p = q} constraint there exists an edge {\tt q $\to$ p}.
	\item All complex constraints (containing dereferences or field offsets) are found and attached to their corresponding nodes in the constraint graph.
	\item Strongly connected components in the graph are found and contracted.
	\item All nodes are put into a worklist.
	\item A node is taken from the worklist, all complex constraints are processed (possibly adding more copy edges to the graph) and its set is propagated alog the copy edges. All nodes modified by this operation are put into the worklist.
	\label{gcc-aa-propstep}
	\item Repeat step \ref{gcc-aa-propstep}. while there are elements in the worklist.
\end{enumerate}
\end{tcolorbox}
\caption{GCC implementation of Andersen's algorithm}
\end{figure}

There are a few tricks to this implementation. Most importantly it actually has
two modes, one for intraprocedural points-to analysis (PTA), one for
interprocedural points-to analysis (IPA PTA). This complicates the development.
We want to keep the intraprocedural analysis as it actually performs well, and
modify the interprodecural version which has performance issues as discusses in
earlier chapters.

\subsection{Improving the implementation}

In this work, we have temporarily duplicated the code into {\tt
ik-structlias.c}, which was then modified only to work only on the interprocedural
version. Though this is not a good practice in general, it brings some good
opportunities.  We can run the original and modified IPA PTA algorithm in a
single execution and directly compare the results. Furthermore, while
benchmarking only the PTA code changes, the rest of the compiler is identical.
The most important reason is to avoid case separation in each function, as we
will not only need to pass different data types, but some operations are no
longer supported, and some operations should be used with greater care than
they are now. This is the case for bitmap difference which is used to discover
changed bit in a bitmap, and enumeration of elements, which is used a few times
during propagation, but can be avoided in some cases. After inspecting the
code, it became clear that the two algorithms need to be separated.

To split the code, a few modifications has been done. Common functions were
marked static and renamed to avoid confusion. New query functions were added to
{\tt tree-ssa-alias.c}, which ask both the original and the new IPA PTA oracle
if the data is available. A new pass has been created, called {\tt kpta} and options added to control it:

\begin{itemize}
	\item {\tt -fipa-kpta} is an analog to {\tt -fipa-pta} and instructs the compiler to run the new IPA PTA algorithm during LTO phase, just after the original IPA PTA pass (if enabled).
	\item {\tt --param kpta-bloomap-size=n} instructs the compiler to use a bloomap of specific size.
	\item {\tt --param kpta-bloomap-precision=p} instructs the compiler to use a bloomap of specific precision. The value passed is inverted precision in percent, so a value of $100$ will result in a precision of $1\%$.
\end{itemize}


\subsection{Integrating Bloomaps}

Integrating bloomaps was relatively straightforward. A few steps were necessary, as some operations do not map well to Bloomap operations.

\begin{itemize}
	\item The main loop in {\tt solve\_graph()} keeps two solution sets, a
		current one and one from previous iteration. When new elements are to
		be propagated (via complex constraints), only the difference is
		examined to make changes. This is a nice optimization for classical
		bitmaps, Bloomap has no easy method to list difference and would have
		to be enumerated. This is unavoidable, but we do not store two bloomaps
		just for this optimization.
	\item Due to historical reasons, when all the sets were computed, they were
		translated from internal variable ids to {\tt DECL\_UID}s. The
		rationale for this was to avoid having to fixup when a functions was
		later inlined. This is unnecessary in recent versions, as all inlining
		happens before IPA PTA pass. It would also introduce additional errors,
		as the bloomap would have to be enumerated and false positives would be
		inserted into a new bloomap. This step was replaced by translating {\tt
		DECL\_UID}s to internal ids during each query.
	\item Bitmap deduplication was removed. After translating the ids, bitmaps
		were hashed and deduplicated via a list of shared bitmaps. This
		deduplication results in a less bitmaps, but is not worth the effort
		for bloomaps, as they are already very space efficient.
	\item During structure field expansion, the set is enumerated a few times.
		A set is temporarily converted to classical bitmap in this step to
		avoid unnecessary work during bloomap enumeration.
\end{itemize}

\section{Performance evaluation}

\begin{figure}[h!]
	\label{figure-firefox-ipa-kpta}
	\centering
	\includegraphics{./graphs/firefox-ipa-kpta/firefox-ipa-kpta.pdf}
	\caption{Building libxul.so with {\tt -fipa-kpta -flto=8}}
\end{figure}


\section{Future work}


\chapter*{Conclusion}
\addcontentsline{toc}{chapter}{Conclusion}

In this work we have identified bitmaps as one of the most used data
structure in GCC and one of its biggest users, the interprocedural
points-to analysis. We have enhanced this algorithm with a new data structure
based on Bloom filters, the Bloomap. In the link-time optimization of Firefox,
we decreased memory usage from 13 GB per process to 3 GB per process during
link-time optimization phase, and build time  has been decreased from 18 hours
with the old pass to around 16 minutes with the improved pass. This is a major
improvement and enables us to analyze programs that could not have been
analyzed before without excessive resource use.

To our knowledge it is the first open-source implementation able to compute
interprocedural points-to analysis for projects like Firefox using reasonable
resources.  The code currently exists as a patch to GCC. It works well in
production environment and has been checked to give conservatively correct
results. We work toward including the code in mainline GCC.


%\appendix
\chapter{GCC cookbook}

Working on GCC is not easy, as it's not a program like others. It's a suite of
compilers\footnote{GCC stands for GNU Compiler Collection}, but contains some
other libraries (for example, the C++ standard library). Moreover, to test our
results, we need to invoke the newly compiled compiler, instead of the system
one. This appendix contains some useful tips, that should be enough to test all
the provided code and replicate the results.

\section{Compiling GCC}

As any other package utilizing {\tt autotools}, GCC can be compiled by a simple
{\tt ./configure \&\& make \&\& make install}. Besides the usual needs, as
configuring a proper prefix (so it won't overwrite our system files), this
approach has several other limitations. For the most part, it takes way too
long to build, as it builds all the language frontends and bootstraps itself.
The process of bootstrapping serves two main function, to test the new compiler
and to provide a more optimized version (for example, if the compiler on host
system was too old to support some optimizations).

Both of these features are good and useful in most cases. None of them is
useful in development and debugging. Let's see why:

\paragraph{Bootstrap.} It compiles GCC three times in a row, each time with the
previous version. This not only takes time, but makes debugging harder, as we
now have the additional need to identify, if a given manifestation is a direct
result of our bug, or a result of miscompiled first (or later) stage.

\paragraph{Language frontends.} GCC itself is written mostly in C/C++, and
those are the only languages required to build a working compiler. Building
frontends that won't be tested is a waste of time.

\paragraph{Multilib.} Provides support for running 32bit code on 64bit
machines. We usually won't need it for development.

After taking all of this into account, let's see how our build will look:

\begin{verbatim}
mkdir obj-build; cd obj-build
../configure --prefix=$HOME/gcc/dev --enable-languages=c++,c \
   --disable-bootstrap --enable-maintainer-mode --disable-multilib
make
make install
\end{verbatim}

A build out of repository root is recommended (we used obj-build), maintainer
is required if configuration for autotools or automake was changed (it was in
this project).

After this, we have a working compiler ready to use, in {\tt \$HOME/gcc/dev}.
This path can be inserted into the PATH system variable, to prefer our new
development compiler.

One of the attachments is a handy script that handles all of this, and provides
not only easy to use interface, but also reproducibility of results.

\section{Runtime libraries}

As mentioned above, the runtime library is part of the GCC suite. It's usually
built during the first build, and never touch again unless it's sources were
changed. This gives us an opportunity to cheat little: building the runtime
with a known working compiler, and commit the changes under scrutiny after the
runtime is built. This gives us the assurance that bugs we are seeing are
directly caused by our compiler, not by a miscompilation of runtime library.

One of the usual symptoms follows: after a unsuccessful modification the
changes are reset and a known working version is checked-out, the new compiler
may seem to be still malfunctioning. This may be caused by a miscompiled
runtime, built with the (buggy) experimental compiler.


%%% Bibliography
%\include{bibliography}
\printbibliography

%%% Figures used in the thesis (consider if this is needed)
\listoffigures

%%% Tables used in the thesis (consider if this is needed)
%%% In mathematical theses, it could be better to move the list of tables to the beginning of the thesis.
\listoftables

%%% Abbreviations used in the thesis, if any, including their explanation
%%% In mathematical theses, it could be better to move the list of abbreviations to the beginning of the thesis.
\chapwithtoc{List of Abbreviations}
\begin{tabular}{ll}
BDD		& Binary Decision Diagram\\
CLA		& Compile Link Analyze\\
DSA		& Data Structure Analysis \\
GCC		& GNU Compiler Collection \\
GNU		& GNU is Not Unix\\
GNU GPL	& GNU General Public License \\
IBLT	& Invertible Bloom Lookup Table\\
ICC		& Intel C++ Compiler\\
IL		& Intermediary Language\\
IPA		& InterProcedural Analysis\\
JIT		& Just In Time\\
LLVM	& Low Level Virtual Machine\\
LTO		& Link-Time Optimization\\
LTRANS	& Local TRANSformation \\
PTA		& Points-To Analysis \\
RTL		& Register Transfer Language \\
TBAA	& Type Based Alias Analysis \\
WPA		& Whole Program Analysis \\
\end{tabular}


%%% Attachments to the master thesis, if any. Each attachment must be
%%% referred to at least once from the text of the thesis. Attachments
%%% are numbered.
%%%
%%% The printed version should preferably contain attachments, which can be
%%% read (additional tables and charts, supplementary text, examples of
%%% program output, etc.). The electronic version is more suited for attachments
%%% which will likely be used in an electronic form rather than read (program
%%% source code, data files, interactive charts, etc.). Electronic attachments
%%% should be uploaded to SIS and optionally also included in the thesis on a~CD/DVD.
\chapwithtoc{Attachments}
\label{attachments}

\noindent {\tt Bloomap.zip} \\ 
\indent Implementation of Bloomaps in C++. Also available online on github: \goodbreak \url{http://github.com/Krakonos/Bloomap} \\

\noindent {\tt gcc-ipa-kpta.diff} \\
\indent Patch to GCC implementing IPA KPTA. Also available online on github: \goodbreak \url{http://github.com/Krakonos/kgcc} \\

\noindent {\tt cgstat}				 \\
\indent A tool for measuring resource usage using Linux Control Groups. \\

\noindent {\tt thesis.pdf} \\
\indent Digital version of this thesis.


\openright
\end{document}

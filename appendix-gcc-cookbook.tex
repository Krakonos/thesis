\appendix
\chapter{GCC cookbook}

Working on GCC is not easy, as it's not a program like others. It's a suite of
compilers\footnote{GCC stands for GNU Compiler Collection}, but contains some
other libraries (for example, the C++ standard library). Moreover, to test our
results, we need to invoke the newly compiled compiler, instead of the system
one. This appendix contains some useful tips, that should be enough to test all
the provided code and replicate the results.

\section{Compiling GCC}

As any other package utilizing {\tt autotools}, GCC can be compiled by a simple
{\tt ./configure \&\& make \&\& make install}. Besides the usual needs, as
configuring a proper prefix (so it won't overwrite our system files), this
approach has several other limitations. For the most part, it takes way too
long to build, as it builds all the language frontends and bootstraps itself.
The process of bootstrapping serves two main function, to test the new compiler
and to provide a more optimized version (for example, if the compiler on host
system was too old to support some optimizations).

Both of these features are good and useful in most cases. None of them is
useful in development and debugging. Let's see why:

\paragraph{Bootstrap.} It compiles GCC three times in a row, each time with the
previous version. This not only takes time, but makes debugging harder, as we
now have the additional need to identify, if a given manifestation is a direct
result of our bug, or a result of miscompiled first (or later) stage.

\paragraph{Language frontends.} GCC itself is written mostly in C/C++, and
those are the only languages required to build a working compiler. Building
frontends that won't be tested is a waste of time.

\paragraph{Multilib.} Provides support for running 32bit code on 64bit
machines. We usually won't need it for development.

After taking all of this into account, let's see how our build will look:

\begin{verbatim}
mkdir obj-build; cd obj-build
../configure --prefix=$HOME/gcc/dev --enable-languages=c++,c \
   --disable-bootstrap --enable-maintainer-mode --disable-multilib
make
make install
\end{verbatim}

A build out of repository root is recommended (we used obj-build), maintainer
is required if configuration for autotools or automake was changed (it was in
this project).

After this, we have a working compiler ready to use, in {\tt \$HOME/gcc/dev}.
This path can be inserted into the PATH system variable, to prefer our new
development compiler.

One of the attachments is a handy script that handles all of this, and provides
not only easy to use interface, but also reproducibility of results.

\section{Runtime libraries}

As mentioned above, the runtime library is part of the GCC suite. It's usually
built during the first build, and never touch again unless it's sources were
changed. This gives us an opportunity to cheat little: building the runtime
with a known working compiler, and commit the changes under scrutiny after the
runtime is built. This gives us the assurance that bugs we are seeing are
directly caused by our compiler, not by a miscompilation of runtime library.

One of the usual symptoms follows: after a unsuccessful modification the
changes are reset and a known working version is checked-out, the new compiler
may seem to be still malfunctioning. This may be caused by a miscompiled
runtime, built with the (buggy) experimental compiler.

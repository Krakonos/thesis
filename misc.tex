\chapter{Misc stuff -- to be moved elsewhere}
\section{Experimenting with GCC}

As I previously noted, I will be focusing on the GCC compiler suite. In this
section, I will cover the experimental setup, compile a few programs and extract
some practical information on resources needed and anticipated use.

The raw data gained in this chapter will not be published, as they tend to be
rather large. I will however publish source code used to take these
measurements, so anyone can reproduce these results, and perhaps use for
comparison on his/her own work. I will also omit some technical details in this
chapter, but will include them in the Appendix \TODO{link}.

\subsection{The setup}

\subsection{Compiler}

For further measurements, I will be using GCC compiled from branch {\tt
gcc-5-branch} (via github.com mirror, but any up to date repository will
suffice). The reason for this branch is simple: it's relatively fresh branch,
that supports most of the latest features, but will not change during
development and provide stable base for testing, while still receiving bug fixes
for the time being.

This is especially important due to the fact that some newer releases have
trouble compiling software like Firefox, which is essentil for some of my
benchmarks. I could fix those, it isn't worth the trouble to do so, and the best
performing code will be ported to the current master branch.

Also, the GCC is non-bootstrapped, but compiled with a system-installed compiler
of the same version, so there should be little to no performance hit. However,
what comes with a performance hit are the benchmarking outputs themselves,
though I will note the time spent in the benchmarking code if it's significant.

\subsection{Software under test}

I have checked many opensource programs to pick some candidates, that could
serve as test inputs. I had a few requirements, to make testing straightforward
and reproducible:

\begin{itemize}
	\item Written in C++ (preferred) or C.
	\item A good compatibility with current GCC versions (5.x and 6.x)
	\item Flexible and robust build system.
	\item Mid to large codebase.
	\item Not too modular.
\end{itemize}

It's suprisingly difficult to find projects that fit all of those, but all of
them are very important. Among others, I first considered well-known projects
like Firefox, GIMP, Inkscape, MySQL and SQLite. I ruled out Inkscape and MySQL




\begin{figure}[h!]
	\label{figure-firefox-ipa-kpta}
	\centering
	\includegraphics{./graphs/firefox-ipa-kpta/firefox-ipa-kpta.pdf}
	\caption{Building libxul.so with {\tt -fipa-kpta -flto=8}}
\end{figure}

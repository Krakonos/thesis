\chapter{Alias analysis}

The goal of alias analysis is to deduce if given (memory) objects can be reached
via two distinct "paths". In the language C, the object would be a memory
location and the path would be a pointer to memory. This is true even for
languages utilizing references or managed pointer, but for the simplicity, I
will only work with pointers and assume language C. The same principle could be
extended to other languages, as well as most assemblies.

We say two pointers are aliasing if they point to a same memory location. Let's
look at a simple example, the notoric function memcpy():

\begin{verbatim}
void memcpy(char *dest, const char *src, size_t n) {
    char *dest_c = (char *) dest;
    char *src_c = (char *) src;
    for (int i = 0; i < n; i++)
        dest_c[i] = src_c[i];
}
\end{verbatim}

The function would benefit from the alias information of {\it dest} and {\it
src}. If we could prove the pointers always point to the same place, the
function could just be skipped. On the other hand if we could prove the pointers
never point to the same place in memory, we could do something clever, like
moving more bytes at a time, without having to worry we could overwrite the
source memory.

Unfortunately, neither of these cases are usually true, or it's hard to prove
it. The glibc function memcpy solves this by specifying the memory locations
can't overlap. This enables efficient implementations, and let's the programmer
worry about overlapping memory.

We could find many more examples like this, and even more of simpler ones.

As we can see in the example above, it's not easy to show if the pointers alias
or not. In this case, it's due to the fact that they are function parameters,
and we would have to examine all possible parameters this function is ever
supplied, and decided if they can alias. In some cases, we just don't have
enough information, so the correct answer is a conservative one: may alias.

\subsection{Problem variations: }

As noted above, the problem still isn't exactly defined. We could analyze some
relations at compile time, but more complex situations can't be decided before
the program is running on a specific input. It may be possible to check for
aliasing during runtime, especially in managed languages like Java. In this
work, I will be only looking at compile-time alias analysis.

\subsubsection{Flow and context insensitive}

In it's simplest form, we might try to solve the alias analysis problem without
any regard to control flow and context. This basically means we will traverse
all the functions in a program, solve them individually, and in each function,
we would ignore all constrol flow statements.

This is relatively easy to do and is currently being done in gcc.\footnote{The
Alias Oracle resides in the tree-ssa-alias.c and tree-ssa-structalias.c}

\subsubsection{Flow sensitive}

By ignoring control flow we have significantly simplified the problem, as
there are some rather complicated conditionals we have avoided. However, we have
missed some easy ways to optimize. Let us look at another example:

\begin{verbatim}
int a, b, n, *x, *y;
if (a > b) {
    x = &a;
    y = &b;
} else {
    x = &b;
    y = &a;
}
memcpy(x,y,n);
\end{verbatim}

Let us denote by {\it PT(\tt x\it)} the set of memory locations {\tt x} can
point to. If we ignore the control flow, we can see that {\it PT(\tt x\it) = PT(\tt y\it)
= \{\tt a\it, \tt b\it\}}. 

If we take the conditional into consideration, even without knowing anything about
$a$ and $b$, we can see that $x$ and $y$ never aliases in this code.

Moreover, we can have specific alias information if we know where the alias
question is asked. That is, if we only care about the {\tt else} block, we can
derive the exact $PT$ sets and alias information. But in the memcpy call, we
only know the alias information, not the exact PT sets (as there are two
possibilites). If we want to work further, we can, but have to represent both
branches.

One can easily see that the amount of information needed is going to grow
exponentially.

\subsubsection{Context sensitive}

By context, we usually mean call context, and thus the context sensitive
analysis is only available in the interprocedural case.



\subsection{Problem complexity}

Before we start diving into the problem variations and algorithms let us
consider how complex the problem really is, as we have to manage our
expectations.

Let us consider the following C code, where $X$, $Y$ and $Z$ are some constants and
{\tt x29A} is some evil function.

\begin{verbatim}
void x29A(void **arg1, void *arg2, char *str);

int main() {
	void *a = X;
	void *b = Y;
	x29A(&a, b, Z);
	some_call(a, b)
}
\end{verbatim}

Consider we want to know the exact points-to sets for $a$ and $b$. We now have
two problems, both of them hidden in the {\tt x29A}. Unfortunately, {\tt x29A}
is evil and can do almost anything:

The obvious problem is, that {\tt x29A} might change the pointer $a$. This
requires the analysis of {\tt x29A} which might, or might not be possible. Let's
consider the following simple code for {\tt x29A}:

\begin{verbatim}
void x29A(void **arg1, void *arg2, char *str) {
	execute(str);
	*arg1 = arg2;
}
\end{verbatim}

As the function {\tt execute} has no access to $a$, the set $PT(a)$ can be
either ${X}$ or ${X, b}$, depending on the finiteness of {\tt execute()}.

Even when knowing the comlete code of {\tt execute()} and the contents of {str},
there is now way of knowing if {\tt execute()} ever returns, as it might be
simulator of turing machine and {\tt str} some arbitrary machine and it's input.
This means that solving exact points-to sets is equivalent  to solving halting
problem, which is no good.

\subsubsection{May and Must aliases}

To make our problem tractable we have to give up the exactness. Let us assume we
only want to compute May-alises or Must-aliases, that is a set of objects a
pointer may alias, or decide if two pointers must point to the same object. This
seems considerably easier, as we can now make some assumption. 

Consider the function $x29A$ from the previous example. We can assume it exits
and continues execution as expected. If this were not the case, our analysis
would be void.  This already means we don't have to solve the halting problem,
which is good.  Moreover, if the body of $x29A$ is available, we can potentially
explore all possible execution paths, and check them for assignments to $a$. If
there is at least one path that assigns $Z$ to $a$, put $Z$ into a's PT set. If
all paths necessarily end with $a = \&Z$ or $a = b$, we that $a$ must point to
$Z$ directly, or must alias with $b$ respectively.

Both of these are very useful but must-alias is in many cases repleceable by
value propagation, and we will not be considering it further.

\subsubsection{Single-level pointers}

Let us first analyze the complexity of may-alias in a simple case, where only
single level pointers exist. Let us denote by $n$ the number of pointers, by $v$
number of scalar variables and other objects a pointer can point to (functions,
heap locations, ...).

Suppose we run the following intra-procedural flow-insensitive algorithm,
assuming we have all the information needed:

\begin{enumerate}
	\item For each pointer variable $p_i$, let $PT := \{ 0_i \}$, where $0_i$ is
		the initial value of $p_i$.
	\item Propagate points-to sets for each modification of $p_i$.
\label{triv-alg-prop}
	\item If any set was changed in step \ref{triv-alg-prop}, go back to
		\ref{triv-alg-prop}.
\end{enumerate}

We can easily formulate this problem as system of set inequalities where:

\begin{itemize}
	\item $p_i := \&a$ is $a \in p_i$ for $a$ scalar
	\item $p_i := p_j$ is $p_j \subseteq p_i$
\end{itemize}

As there are only single level pointers, we are not allowed to take address of a
pointer, and dereference always results in a scalar, which does not change any
pointer.

Now it's clear that when the algorithm exits, the solution is conservatively
correct. Also, if $n$ and $v$ are finite, it will finish after at most $n \cdot
v$ steps, as in each step at least one set will increase in size, and every set
can have at most $v$ elements in.
This gives us simple (not necessarily efficient) polynomial-time algorithm.

Of course there is a problem that in practice we  do not have all the information.
That may include external functions (possibly with side-effects), dynamically
allocated memory and more obscure, possibly language-specific, features as for
example pointer arithmetic. We will ignore these now for simplicity, and
deal with them later. \TODO{Reference.}

\subsubsection{Multi-level pointers}

Things start to be difficult when multi-level pointers come in play. 
\TODO{Proof.}

\section{Known algorithms and approaches}

During the years a few algorithmic approaches have been developed. Here is a
quick overview of the strengths and weaknesses of those approaches.

\subsection{Type based algorithms (TBAA)}

As the name suggests, they infer aliasing information from types and language
rules. This is is especially useful in languges with strict aliasing rules, but
rarely is more difficult than comparing those types using a set of rules. 

\subsection{Andersen's algorithm}

First published by Lars Ole Andersen \cite{Andersen94}, it is an {\it
inclusion-based} algorithm is based on direct mathematical representation of
aliases as points-to sets. That is, a points-to set for a given pointer $p$ is a
set $S_p$ containing all locations pointer $p$ can point to.  Further
expressions are then translated into set inequalities:

\begin{itemize}
	\item $p_i = \&a \quad \to \quad a \in p_i$
	\item $p_i = p_j \quad \to \quad p_j \subseteq p_i$ 
	\item $p_i = *p_j \quad \to \quad \forall p_k \in p_j : p_k \subseteq p_i$
\end{itemize}

The solution to this set of inequalities gives the may-alias points-to
solution. This representation also offers a simple algorithm for iterative
solving, but no guarantee about memory requirements of time of completion.

\subsection{Steensgaard's algorithm}

Developed by Bjarne Steensgaard \cite{Steensgaard96}, it is  a {\it
unification-based}, similar to Andersen's, but uses unification instead of
subset contraints. It only needs to partition pointers into equavalence
classes, which can be done in almost linear time. This leads to a very fast,
well scalable algorithm, that sacrifices some precision.

There is very little research utilizing unification-based algorithms, as it is
believed to be patented by Microsoft, and was removed from LLVM in 2006
\cite{LLVM:DSA:Remove}.


\section{Data structures}

Especially in Adersen-style algorithms, the structure used is very important, as
it needs to efficiently store potentially large sets and process set operations
on them (mostly {\tt UNION} and {\tt INTERSECTION}).

In this section we will review several straightforward implementation and the
use of bloom filters and binary decision diagrams.

\TODO{Compare some naive representations}

\subsection{Bloom filters}

The use of Bloom filters was first proposed by Nasre et. al \cite{nasre2009}.
They are very space efficient and perform well on certain operations, as is
{\tt QUERY} and {\tt UNION}. Some implementation can also perform {\tt
INTERSECTION}, but with decresed precision. The complete lack of {\tt
ENUMERATION} was worked around by introducting multiple dimensions for
multi-level pointers. In this scheme, a pointer could be easily dereferenced
upto a constant depth and after that, the algorithm answers conservatively.

We will revisit the use of Bloom filters in later chapters.

\subsection{Binary decision diagrams based algorithms}

A Binary decision diagram (BDD) is a data structure used to represent boolean
functions. It can be easily extended to represet relations by encoding
characteristic function of given relation, and the complete alias information as
well. Multiple algorithms based on the BDDs were developed \cite{whaley2004},
\cite{bddbddb}, but most of them lack public and usable code for further
development.


\section{Current state in compilers}

There are not many modern compilers with open code that can be examined and improved
upon. One of the players is GCC, that has been around since 1985
(1.x release was in 1991) and is the most widely used open source compiler
today. The younger competitor is LLVM/Clang, first released in 2003. It's
written in C++, is supported by Apple since 2005, and due to it's age has much
mure modern design, and is generally deemed to be easier to extend and work
with.

There are much more compilers available, but most of them are proprietary or not
maintained.

Also, it is very hard to compare many of the published results, as the
implementations are not public, and mostly implemented for compilers that are
unable to keep up with current C/C++ standards and successfully build modern
(and big) projects.

A lot of researchers also focus on Java compiles and algorithms, and though many
techniques can be used for C and C++, Java is very different language, in that
it has JIT\footnote{Just In Time} compiler, and does not have pointers in the
classic sense, only references. 

\subsubsection{LLVM/Clang}

As LLVM is very modular, it contains multiple alias analysis passes. 

\begin{itemize}
	\item {\bf -basic-aa} pass, providing local alias information using many
		language-specific facts.
	\item {\bf -scev-aa} pass, translation queries into Scalar Evolution queries
		\TODO{WTF?}
\end{itemize}

Additionally, there are three passes in {\tt poolalloc} package:

\begin{itemize}
	\item {\bf -globalsmodref-aa} pass, providing context-sensitive alias
		information for global variables
	\item {\bf -steens-aa} pass, implementing Steensgaard's algorithm.
	\item {\bf -ds-aa} pass, implementing unification-based Data Structure
		analysis, providing context and field sensitive alias information.
\end{itemize}

\TODO{Are they interprocedural? steens probably is, but needs checking; more
info on this here: http://llvm.org/docs/AliasAnalysis.html}

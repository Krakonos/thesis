\chapter{Alias Analysis}

The goal of alias analysis is to enable optimizations across memory operations.
It is used by other compiler components to disambiguate accesses to memory
locations, enabling many optimizations. We discuss most commonly used decision
methods, with focus on points-to analysis.

\section{Common alias analysis methods}

In the C Language, a memory location is usually accessed by its name or via a
pointer. Disambiguating two accesses is necessary for many optimizations, but missing an
alias might result in incorrect code generated. See example in Figure
\ref{alias-example1}. The second assignment to {\tt b} might seem redundant, as
{\tt a} could not have changed.  However, it is true only if the call to {\tt
some\_fn} did not change variable {\tt b}.

\begin{figure}[h!]
\begin{tcolorbox}
\begin{lstlisting}{language=c,tabsize=2}
void set_call_set(void) {
	int a,b;
	[...]
	b = a + 1;
	some_fn(a, &b);
	b = a + 1;
	[...]
}
\end{lstlisting}
\end{tcolorbox}
\caption{Example of the importance of alias information}
\label{alias-example1}
\end{figure}

Accurately disambiguating memory references may be arbitrarily complex. Many
optimizations will be possible even with a minimal aliasing information.
Consider example in Figure \ref{alias-example2}. The loop seems to write zeroes
into an array {\tt a}. This is true if the pointer dereference can not
change the pointer itself. Fortunately we do not need to examine any code not
shown in the example. The C standard prohibits the dereference of {\tt float*}
to modify {\tt float*} itself \cite{isoc}. This method is called {\it Type-Based
Alias Analysis} (TBAA).

\begin{figure}[h!]
\begin{tcolorbox}
\begin{lstlisting}{language=c,tabsize=2}
void fill_floats(void) {
	float* a;
	[...]
	for (int i = 0; i < 10; i++)
		*(a++) = 0;
}
\end{lstlisting}
\end{tcolorbox}
	\caption{Example of the importance of alias information}
	\label{alias-example2}
\end{figure}

Alias analysis can not be solved accurately.  We distinguish between may-alias
and must-alias information, enabling us to give conservative answers.  May-alias
information indicates that the access may alias on at least one path
in the control flow graph. On the other hand, must-alias information requires
alias in all possible paths. Consider example in Figure
\ref{alias-example-maymust}. The information that ``{\tt p} points-to {\tt x} or
'{\tt y}' is an example of may-alias, as it depends on the condition taken.
The information ``{\tt q} points-to {\tt x}'' is an example of must-alias, as it
holds on all paths in the example. Notice that must-alias always returns a
single element, may-alias usually returns larger points-to sets.

\begin{figure}[h!]
\begin{tcolorbox}
\begin{lstlisting}{language=c++,tabsize=2}
void fill_floats(void) {
	int *p,*q;
	int x,y;
	[...]
	q = &x;
	if (x > y) {
		p = &x;
		[...]
	} else {
		p = &y;
		[...]
	}
	*p = 0;
	[...]
	
}
\end{lstlisting}
\end{tcolorbox}
\caption{Example of may and must-aliasing}
\label{alias-example-maymust}
\end{figure}

Alias analysis can be seen as a separate module of a compiler, which is accessed
by optimization passes by the means of {\it alias oracle}. This usually is a
function that given two memory accesses in a program answers if they can access
the same memory location. The answer can be {\it yes}, {\it no} or {\it maybe}.
A single oracle can apply multiple algorithms to determine the answer. The
following three oracles are most often used:

\label{sec-tbaa}
\paragraph{Type Based Alias Analysis} (TBAA) infers aliasing information from
types and language-specific rules, as seen in Figure \ref{alias-example2} for
the C Language \cite{isoc}. This method is
very fast, as it only needs to inspect the types in question. For this reason,
it is usually asked first and is able to distinguish many cases by itself.

\subsection{Base and offset analysis}
\label{sec-baseoffset}

In the case of structures or arrays, the access is composed of base pointer and
an offset. The offset information might not be complete, but sometimes the range
for offset is known. If the bases are distinct memory locations, the accesses do
not alias.  If the bases are provably the same memory locations, the offets can
be compared to see if the accesses can alias.

\subsection{Points-to analysis}

If a memory access cannot be disambiguated by any simpler rule, points-to sets
must be analyzes and evaluated. A {\it points-to set} for a variable (pointer)
is a set of memory locations the variable can be used to access. For example, a
simple non-pointer variable can only be used to access itself (access by name).
A pointer variable can be used to access other memory locations of which the
address was taken. To disambiguate memory accesses the points-to sets have to be
compared and if their intersection is empty, it is safe to assume they do not
alias. If their intersection is non-empty, or some of the sets could not be
computed, we must assume they do intersect to preserve correctness.

Compared to type based and base and offset analysis, points-to analysis is a
time-consuming process and will be a focus of this chapter.


\section{Points-to analysis}

\TODO{Uvod}

It is also useful to distinguish between flow sensitivity, context sensitivity
and their combinatins.

A {\it flow-sensitive} algorithm computes the alias information with regard to control
flow. In the example in Figure \ref{alias-example-maymust} it would notice the
different branches of {\tt if} and provide information that ``{\tt p} points to
{\tt x} in the {\tt if} branch'' and similarly for the {\tt else} branch.

A {\it flow-insensitive} algorithm computes alias information without any regard
to control flow. In the same example it would just output ``{\tt p} may point-to
{\tt x} or {\tt y}''.

Also notice that the flow-sensitive algorithm gave us must-alias
information, while the flow-insensitive only gave may-alias information.
This is to be expected, as the flow-insensitive variation would require only
single assignment to the pointer (at initialization) to output must-alias
information.

Context sensitivity is a similar problem to flow sensitivity but in
intraprocedural case. While flow sensitivity relies on control flow graph inside
a single function, context sensitivity is based on callgraph. 

Before we continue further, let us formally define the various alias-analysis
types. See \cite{muchnick1997advanced}.

\paragraph{Definition} {\it Flow-insensitive may-alias information} is a binary
relation $A_{\mathrm{FinMay}} \subseteq \Var \times \Var $ on the variables. A pair
$(x,y)$ is in the relation if $x$ and $y$ can refer to the same
memory location, possibly at a different place in the program, or at a different
time during execution. This relation is symmetric, but is not transitive.

\paragraph{Definition} {\it Flow-insensitive must-alias information} is a binary
relation $A_{\mathrm{FinMust}} \subseteq \Var \times \Var$ on the variables. A pair
$(x,y)$ is in the relation if and only if $x$ and $y$ always refer to the same
memory location during the program execution. This relation is symmetrict, but
also transitive. 

The flow-sensitive case is a more complicated, and can be examined both as a
relation or function.

\paragraph{Definition} {\it Flow-sensitive may-alias information} is a ternary
relation $A_{\mathrm{FseMay}} \subseteq \Var \times \Var \times \Loc$ on the variables
and program locations. A triplet $(x,y,p)$ is in the relation if $x$
and $y$ can refer to the same memory location at the point $p$ in program
execution.

\paragraph{Definition} {\it Flow-sensitive must-alias information} is a ternary
relation $A_{\mathrm{FseMust}} \subseteq \Var \times \Var \times \Loc$ on the variables
and program locations. A triplet $(x,y,p)$ is in the relation if and only if $x$
and $y$ always refer to the same memory location at the point $p$ in program,
regardless of what the memory location is.

A similar definition could be used for context sensitivity, adding call context
to the relation as well, or encoding it in the location. The specifics depend on
the definition of context, as there are multiple possiblites. A context could be
just a call site, or a path in callgraph from the entry point, possibly only to
a limited depth.

\subsection{Problem complexity}

It is useful to know how difficult the problem of points-to analysis is. In this
section we will review previous results showing the theoretical bounds for
different problem variants.

The earliest classification is from Landi \cite{Landi1991}, who proves that
computing flow-sensitive may- and must-alias information in the presence of
single level pointers can be done in polynomial time. By adding more levels of
indirection, as is common in most languages, the problem becomes NP-hard.

Later Horwitz \cite{Horwitz1997}  proved that precise flow-insensitive alias
analysis is NP-hard with only scalar variables and no heap allocations, though
the result assumes unrestricted pointer dereference.

Chakaravarthy \cite{ptcomp} proved that when heap allocations are allowed the
problem becomes undecidable, even if all the variables are scalar. The same
articles also proves that the flow-insensitive variant is in P, if the
variables are further restricted to well-defined types\footnote{Known type and
number of indirections}. Although this is not always the case, it gives us hope
that a successful alias analysis could be performed on a well-formed program.

\TODO{Praxe}


\subsection{Known algorithms and approaches}

During the years, only a few algorithms have been developed and because alias
analysis is a typical dataflow problem, there is little reason to expect a
practical but fundamentally different algorithm.

\subsubsection{Andersen's algorithm}

First published by Lars Ole Andersen \cite{Andersen94}, it is an {\it
inclusion-based} algorithm is based on direct mathematical representation of
aliases as points-to sets. That is, a points-to set for a given pointer $p$ is a
set $S_p$ containing all locations pointer $p$ can point to.  Further
expressions are then translated into set inequalities:

\begin{align}
	\label{aa-init}
	p_i = \&a \quad &\to \quad a \in p_i \\
	\label{aa-prop}
	p_i = p_j \quad &\to \quad p_j \subseteq p_i \\
	\label{aa-deref}
	p_i = *p_j \quad &\to \quad \forall p_k \in p_j : p_k \subseteq p_i
\end{align}

The structure of proposed Andersen's flow-insensitive algorithm is shown in Figure
\ref{figure-andersen}.

\begin{figure}[h!]
\begin{tcolorbox}
\begin{enumerate}
	\item Initialize variables using \ref{aa-init}.
	\item Build a propagation graph using \ref{aa-prop} and \ref{aa-deref}
		with variables as vertices, propagations along edges.
	\item Find strongly connected components in the grah and merge them into a single node.
	\item Mark every node as changed.
	\item For every changed node, reset its changed status, propagate the change
		along edges and mark nodes as changed if they were modified.
	\label{aa-propstep} 
	\item Repeat step \ref{aa-propstep}. until no node is marked as changed.
\end{enumerate}
\end{tcolorbox}
\caption{Andersen's algorithm}
\label{figure-andersen}
\end{figure}

\subsubsection{Steensgaard's algorithm}

Developed by Bjarne Steensgaard \cite{Steensgaard96}, it is  a {\it
unification-based}, similar to Andersen's, but uses unification instead of
subset contraints. It only needs to partition pointers into equavalence
classes, which can be done in almost linear time. This leads to a very fast and
scalable algorithm, sacrificing some precision.

The unification-based algorithms are less used, as the method is
believed to be patented by Microsoft\cite{patent:steensgaard}. It was
implemented in LLVM, but later removed in 2006 \cite{LLVM:DSA:Remove} due to
patent concerns. We expect this to change, as the patent has just expired while
writing this thesis, in September 2016.

\subsection{Further improvements}

Steensgaard's algorithm can use Union-Find data structure for the unification,
which is already extremely efficient \cite{Tarjan1975}. Andersen's algorithm has
to deal with sets, and the choice of data structure for set management is
harder. Two major improvements have been proposed to date, though none of them
have been implemented in a production compiler.

\subsubsection{Bloom filters}

The use of Bloom filters was first proposed by Nasre et. al \cite{nasre2009}.
They are very space efficient and perform well on certain operations, as is
query and union. Some implementations can also perform interesection, but with
decresed precision. The complete lack of the ability to enumerate elements was
worked around by introducting multiple dimensions for multi-level pointers. In
this scheme, a pointer could be easily dereferenced upto a constant depth and
after that, the algorithm answers conservatively.

We will revisit the use of Bloom filters in later chapters.

\subsubsection{Binary decision diagrams based algorithms}

A Binary decision diagram (BDD) is a data structure used to represent boolean
functions. It can be easily extended to represet relations by encoding
characteristic function of given relation, and the complete alias information as
well. Multiple algorithms based on the BDDs were developed \cite{whaley2004},
\cite{bddbddb}, but most of them lack public and usable code for further
development. The major issue with the use of BDDs is that they heavily rely on
the correct variable ordering. Choosing wrong ordering quickly results in
size explosion and speed decrease. However the BDD approach seems promising for
loss-less representations.

\section{Current state in compilers}

There are not many modern compilers with open code that can be examined and improved
upon. One of the players is GCC, that has been around since 1985
(1.x release was in 1991) and is the most widely used open source compiler
today. The younger competitor is LLVM/Clang, first released in 2003. It is
written in C++, is supported by Apple since 2005, and due to its age has much
mure modern design, and is generally deemed to be easier to extend and work
with. Other competitor is Open64, which lacks community support, but is still
being developed by some groups.

Many researchers also focus on Java compiles and algorithms, and though many
techniques can be used for C and C++, Java is very different language, in that
it has JIT\footnote{Just In Time} compiler, and does not have pointers in the
classic sense, only references, which simplifies some cases.

There are much more compilers available, but most of them are proprietary or not
maintained, as for example the Intel C++ Compiler (ICC) and VisualC++.

It is very hard to compare many of the published results, as the
implementations are not public, and mostly implemented for compilers that are
unable to keep up with current C/C++ standards and successfully build modern
(and big) projects. Many of the results are computed outside the compiler and
never tested. But even if they were, there is no simple metric that could be
used for comparison. The results rely on previous optimization passes,
constraint generator, chosen granularity (wether to consider structure members or
arrays) and finally queries asked by the compiler in later optimization phases.

\subsubsection{GCC}

GCC has a good support for TBAA and Base-offset analysis, intraprocedural
points-to analysis, but lacks a good interprocedural points-to analysis. We
discuss details in later chapters.

\subsubsection{LLVM/Clang}

As LLVM is very modular, it contains multiple alias analysis passes. In the core
package there is {\tt -basic-aa} pass, providing local alias information using
many language-specific facts. It is similar to GCC's TBAA and Base-offset
analysis.

Additionally, there are three passes in {\tt poolalloc} package, the {\tt
-globalsmodref-aa} pass, providing context-sensitive alias information for
global variables (similar to GCC's {\tt ipa-reference pass}. It also implements
the Steensgaard's algorithm in the {\tt -steens-aa} pass, and Andersen-style
context and field-sensitive points-to analysis in {\tt -ds-aa}.

\subsubsection{Open64}

A new context-sensitive andersen-style alias analysis has been implemented in
2013-2014 \cite{sui2014}, though the context sensitivity seems to be only
partially implemented.


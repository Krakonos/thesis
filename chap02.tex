\chapter{Vnitřní struktura GCC}

\section{Fáze překladu}

Během překladu dochází k postupným změnám v reprezentaci dat, jak se program
transformuje z jazyka frontendu do stále jednodušších jazyků, dokud nevznikne
požadovaný spustitelný kód. Celkem dojde k několika fázím:

\begin{enumerate}
	\item Front End: Syntaktický analýza jazyka.
	\item Middle End: Vysokoúrovňové transformace a optimalizace.
	\item Back End: Nízkoúrovňové transformace a optimalizace.
\end{enumerate}

Během tohoto procesu je použito několik mezijazyků, které se dokonce mohou někdy
míchat (to je běžné, pokud se právě převádí z vyššího jazyka na nižší). Zde je
základní přehled, některé budou podrobněji rozepsány dále. Všechny používané
jazyky jsou úzce svázány s objekterý, pomocí kterých jsou reprezentovány.

\begin{description}
	\item[GENERIC] Nejvyšší mezijazyk, dovoluje vyjádřit funkci jako strom, a
		může obsahovat prakticky libovolná rozšíření specifická pro jazyk
		front-endu.
	\item[GIMPLE] Striktní verze jazyka GENERIC, speciálně každý výraz již smí
		obsahovat nanejvýš 3 operandy.
	\item[RTL] Nízkoúrovňový jazyk, podobající se spíše strojovým instrukcím
		cílové architektury. Obsahuje mimojiné jména registrů a práci se
		zásobníkem.
\end{description}

\subsection{Front End}

Front-end je v optimálním případě jediná část překladače, která zná informace o
jazyce, jeho syntaxi, a omezeních. Může svoji funkci provádět různě, a taktéž se
tak děje; nezávisle na tom však musí vydat reprezentaci, která je čitelná pro
další fáze překladu. Tou může být přímo GIMPLE, nebo GENERIC, který se na GIMPLE
převede.

\subsection{Middle End: Vysokoúrovňové transformace a optimalizace}

Jakmile je získán od frontendu kód v GIMPLu, provádí se většina optimalizací.
Dojde k rozdělení kódu na bloky a vybudování CFG (Control Flow Graph) mezi nimi,
čištění kódu, propagace konstant a výrazů, zpracování dat z profilování,
optimalizace smyček, alias analýza a mnoho dalších.

\subsection{Back End: Nízkoúrovňové transformace a optimalizace}

Po vykonání všech vysokoúrovňových transformací dojde ke snížení GIMPLE do
jazyka RTL, již za znalosti konkrétní architektury, pro kterou se překládá.
Vysokoúrovňové konstrukty se přeloží prakticky do výsledné podoby -- na
manipulaci s registry, zásobníkem a dostupné operace nad tímto.

Nad touto reprezentací probíhají zejména optimalizace přemisťující fyzicky kód v
paměti, optimalizují skoky a plánují instrukce. Taktéž dochází k alokaci
registrů.

\section{Reprezentace dat}

\section{Optimalizační průchody}

\section{LTO optimalizace}

\section{Current Alias Analysis implementations}

There are several passes in GCC that provide alias informations.


\chapter*{Preface}
\addcontentsline{toc}{chapter}{Preface}

%Dosavadní implementace alias analýzy v GCC je založena na článcích [Efficient
%Field-sensitive pointer analysis for C" a "Ultra-fast Aliasing Analysis using
%CLA: A Million Lines of C Code in a Second".
%
%Pro výpočet analýzy používá constraint-graph, který obsahuje výrazy typu
%Dereference, vzetí adresy, a skalární, nad kterými spočítá (?) tranzitivní
%uzávěr a vydedukuje points-to množiny. 
%
%Toto má mnohé nevýhody, zejména praktickou nepoužitelnost při optimalizacích
%celých programů (? IPA a LTO).
%
%Právě na téma škálovatelnosti se v posledních letech upřely zraky mnohých 
%akademiků i programátorů. Příchod LTO a existence programů obrovských rozměrů, 
%jako je například Firefox či KDE, zapříčinily, že standardní postupy již nejsou 
%použitelné.
%
%Cílem této práce je navrhnout algoritmy a datové struktury, které budou
%dostatečně rychlé a prostorově úsporné, aby byly použitelné při kompilaci
%velkých programů bez potřeby superpočítačů.

As soon as programs started their growth, it became necessity to split them into
functions, and later into compilation units. This shields the programmer from
unnecessary technical details of implementation, and allows him to concentrate on
the actual work. The same limits the scope of the compiler, which can not
optimize beyond unit boundaries.  This is a major disadvantage as the compiler
needs as much information as it can get to generate better code.

For a long time, most compilers worked on separate compilation units, and did not
really care about other units in terms of analysis.  However  thecompilers grow
more capable each year, as does the computing power of consumer-grade machines.
We have reached the point where advanced optimizations could be performed even
on very large programs.

In 2009, the GCC project merged link-time optimizer into version 4.5, which enables
analysis and optimization on scope of all compilation units. This offers new
opportunities for improvement, but also new challenges. Some of the existing
algorithms can easily process whole programs, some have limitations, and some
of them are just too slow and/or memory intesive to be used in production.

The goal of this thesis is to explore current link-time optimization
techniques, identify a bottleneck, and improve upon it. 

This thesis is organized as follows: We introduce compilation and link-time
optimization techniques in the first chapter, introduce Alias Analysis problem
and it's possible solutions in the second chapter, with emphasis on practical
use in current compilers. We improve upon Andersen's inclusion-based algorithm
in the third chapter, by using efficient data-structure derived from Bloom
filters, sacrificing some precision for tractability, and compare the results
with non-approximate solution using the same algorithm. In the last chapter we
provide documentation necessary to reproduce presented results.



%Klasický přístup k překládání programů je založen na separaci kódu tak, že
%jednotlivé překladové jednotky (typicky složené z několika málo souborů) se
%přeloží najednou, a až po přeložení všech jednotek se program slinkuje do
%jednoho výsledného objektu.
%
%Tento přístup šetří zdroje počítače, který překlad provádí: jednak není nikdy
%zapotřebí mít nahraný celý program v paměti, dále je možno paralelizovat a
%překládat jednotlivé jednotky nezávisle. Nadruhou stranu však tento přístup
%zbraňuje překladači provádět optimalizace na rozhraní překladových jednotek.
%
%S tím, jak roste síla počítačů, a zejména dostupné paměti, je možné kompilační
%jednotky dále zvětšovat a pokoušet se o optimalizaci programů jako celku.
%
%Některé projekty se daly vlastní cestou, a tento problém vyřešili
%předzpracováním, které celý zdrojový kód vložilo do jednoho souboru; příklad
%může být SQLite, oblíbená relační SQL databáze, v jednom souboru obsahujícím cca
%370 řádek kódu v jazyce C.
%
%U větších projektů, jako například KDE, známé desktopové prostředí, se však brzy
%stane tento postup neunesitelný, a to zejména díky zdrojům, které jsou zapotřebí
%na překlad programu, a také díky nemožnosti snadno paralelizovat, což v dnešní
%době vícejádrových procesorů je nedostatek zcela zásadní.
%
%Tato bariéra je částečně odstraněna zavedení možnosti vložit do částečně
%přeloženého objektu dostatek metadat na to, aby ve fázi linkování šlo spustit
%některé optimalizační průchody, tzv. LTO (Link Time Optimization).
%
%Některé optimalizační průchody však byly navrženy dosti neefektivně s tím, že
%fungují dobře na jednotlivých funkcích, uspokojivě na menších překladových
%jednotkách, ale již dobře neškálují na celé programy, a to jak potřebným
%výpočetním časem, tak prostorem, a to i o několik řádů.
%
%Cílem této práce je zanalyzovat stav optimalizací, které aktuálně nejsou
%použitelné, navrhnout a implementovat řešení.
%

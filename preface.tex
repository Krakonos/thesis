\chapter*{Preface}
\addcontentsline{toc}{chapter}{Úvod}

%Dosavadní implementace alias analýzy v GCC je založena na článcích [Efficient
%Field-sensitive pointer analysis for C" a "Ultra-fast Aliasing Analysis using
%CLA: A Million Lines of C Code in a Second".
%
%Pro výpočet analýzy používá constraint-graph, který obsahuje výrazy typu
%Dereference, vzetí adresy, a skalární, nad kterými spočítá (?) tranzitivní
%uzávěr a vydedukuje points-to množiny. 
%
%Toto má mnohé nevýhody, zejména praktickou nepoužitelnost při optimalizacích
%celých programů (? IPA a LTO).
%
%Právě na téma škálovatelnosti se v posledních letech upřely zraky mnohých 
%akademiků i programátorů. Příchod LTO a existence programů obrovských rozměrů, 
%jako je například Firefox či KDE, zapříčinily, že standardní postupy již nejsou 
%použitelné.
%
%Cílem této práce je navrhnout algoritmy a datové struktury, které budou
%dostatečně rychlé a prostorově úsporné, aby byly použitelné při kompilaci
%velkých programů bez potřeby superpočítačů.

As soon as programs started their growth, it became necessity to split them into
functions, and later into compilation units. This shields the programmer from
unnecessary technical details of implementation, and allows him to concentrate on
the actual work.

Unfortunately, this is also true for the compiler, which is shielded from the
implementation details. However in this case, the implementation details are not
unnecessary and the compiler could do a better job if it knew.

For a long time, most compilers worked on separate compilation units, and didn't
really care about other units in terms of analysis.

In the recent years, computing power of consumer-grade machines increased to the
point where something could be done about it, even for some bigger programs.

In 2009 the GCC merged LTO into version 4.5, which enables optimizations on the
scope of all compilation units.

The goal is to explore current link-time optimization techniques, identify a
bottleneck, and improve upon it.




I will introduce link-time optimization techniques in the first chapter,
introduce Alias Analysis problem and it's possible solutions in the second
chapter, with emphasis on practical use in current compilers. I will improve
upon Andersen's inclusion-based algorithm in the third chapter, by using
efficient data-structure derived from Bloom filters, sacrificing some precision
for tractability, and compare the results with non-approximate solution using
the same algorithm. In the last chapter, I will compare my implementation with
already available implementations in production compilers with respect to
resource use during compilation and their usability for large programs.


%Klasický přístup k překládání programů je založen na separaci kódu tak, že
%jednotlivé překladové jednotky (typicky složené z několika málo souborů) se
%přeloží najednou, a až po přeložení všech jednotek se program slinkuje do
%jednoho výsledného objektu.
%
%Tento přístup šetří zdroje počítače, který překlad provádí: jednak není nikdy
%zapotřebí mít nahraný celý program v paměti, dále je možno paralelizovat a
%překládat jednotlivé jednotky nezávisle. Nadruhou stranu však tento přístup
%zbraňuje překladači provádět optimalizace na rozhraní překladových jednotek.
%
%S tím, jak roste síla počítačů, a zejména dostupné paměti, je možné kompilační
%jednotky dále zvětšovat a pokoušet se o optimalizaci programů jako celku.
%
%Některé projekty se daly vlastní cestou, a tento problém vyřešili
%předzpracováním, které celý zdrojový kód vložilo do jednoho souboru; příklad
%může být SQLite, oblíbená relační SQL databáze, v jednom souboru obsahujícím cca
%370 řádek kódu v jazyce C.
%
%U větších projektů, jako například KDE, známé desktopové prostředí, se však brzy
%stane tento postup neunesitelný, a to zejména díky zdrojům, které jsou zapotřebí
%na překlad programu, a také díky nemožnosti snadno paralelizovat, což v dnešní
%době vícejádrových procesorů je nedostatek zcela zásadní.
%
%Tato bariéra je částečně odstraněna zavedení možnosti vložit do částečně
%přeloženého objektu dostatek metadat na to, aby ve fázi linkování šlo spustit
%některé optimalizační průchody, tzv. LTO (Link Time Optimization).
%
%Některé optimalizační průchody však byly navrženy dosti neefektivně s tím, že
%fungují dobře na jednotlivých funkcích, uspokojivě na menších překladových
%jednotkách, ale již dobře neškálují na celé programy, a to jak potřebným
%výpočetním časem, tak prostorem, a to i o několik řádů.
%
%Cílem této práce je zanalyzovat stav optimalizací, které aktuálně nejsou
%použitelné, navrhnout a implementovat řešení.
%
